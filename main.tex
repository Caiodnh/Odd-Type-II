\documentclass{amsbook}
\usepackage{basic-preamble}
\overfullrule=2cm

% begin checklist
\usepackage{pifont}
\newcommand{\cmark}{\ding{51}}%
\newcommand{\xmark}{\ding{55}}%

\newcommand{\done}{\rlap{\raisebox{1pt}{\large\hspace{-5pt}\cmark}}}

\newcommand{\cancel}{\rlap{\raisebox{0pt}{\large\hspace{-5pt}\xmark}}}
%\newcommand{\donebox}{\rlap{$\square$}{\raisebox{2pt}{\large\hspace{1pt}\cmark}}%
% end checklist



\newcommand\numberthis{\addtocounter{equation}{1}\tag{\theequation}}

\begin{document}

% ==================================================
% Header
% ==================================================

% Authors
% -------

\author[Caio]{Caio}
\address{Department of Mathematics and Statistics,
	Memorial University of Newfoundland,
St. John's, NL, A1C5S7, Canada}
\email{cdnh22@mun.ca}

% Footnotes
% ---------


% Title
% -----

\date{}

\title{Odd Type II}

% Classification and key words
% ----------------------------

\subjclass[2010]{Primary 17B70; Secondary 16W50, 16W55, 17A70}
\keywords{Graded algebra, associative superalgebra, simple Lie superalgebra, classical Lie superalgebra}

% Abstract
% --------

% Maketitle
% ---------

%\maketitle



\chapter{Overview of $\vphi$-gradings}

This is an overview of what I want done in the section/chapter of $\vphi$-gradings. Each bullet should be expanded to a paragraph/definition/result/proof, roughly in order of appearance in the final text. The goal is to have a better idea of how it will look like.

Some billets are for things that must come before, others are for things the should come later. But these are less detailed. As the real work gets done, the bullets will necessarily change a little.

\section{Before}

\begin{itemize}
        \item The sign in identifying $U \tensor V$ with $V \tensor U$.
    
        \item Side of $R$-linear maps: left or right? How to move from one to another.
    
        \item The dual basis, and how it changes moving the maps to the left (even if $R=\FF$).
    
        \item The supertranspose and definition of super-anti-automorphisms.
        
        \item In general, super-anti-automorphisms don't have order $2$. Indeed, we will see, they only exist for certain choice of $m$ and $n$ (one must be even or both must be equal)
        
        \item Nondegenerate bilinear forms yield super-anti-automorphisms. Supersymmetry implies order $2$.
        
        \item Notation $\FF^{(m|n)}$

\end{itemize}


\section{Actual}

    \begin{itemize}
        \item[\done] Why consider $\vphi$-gradings? They are the type I gradings for Lie superalgebras orthosympletic or peripletic. 
        
        \item[\done] Definition of superalgebra with distinguished map and its homomorphisms. 
        
        \item[\done] Definition of $\Aut (R, \vphi)$. 
        
        \item[\done] Note that if the map is the identity, nothing is new. 
        
        \item[\done] Definition of $\vphi$-grading for any distinguished map $\vphi$. 
        
        \item[\done] Again, $\vphi = \id$ yields nothing new
        
        \item[\done] The case of interest for us is when $\vphi$ is a super-anti-automorphism. In this case the grading restricts to a grading on $\Skew(R, \vphi)$. 
        
        \item[\done] Subsection: Correspondence between $\widehat G$-actions and $\vphi$-gradings. Recall that $\vphi$ has degree $e$ iff it is $\widehat G$-equivariant. Hence homomorphisms of $\vphi$-gradings correspond to maps commuting with $\vphi$.
        
        \item[\done] In particular, $\vphi$-gradings correspond to $\widehat G$-representations by automorphisms of $(R, \vphi)$. (end of subsection)
        
        \item[\done] Subsection: Type II gradings on $A$ as $\vphi$-gradings
        
        \item[\done] Check the beamer about this
        
        \item[\done] New model for $\gl(m|n)$. Take $R = M(m,n) \times M(m,n)$ and consider $\psi$ to be any super-anti-automorphism on $M(m,n)$. Then take $\vphi (x, y) = (\psi\inv (y) , \psi (x) )$ on $R$. It is a superinvolution.
        
        \item[\done] We could take the supertranspose for $\psi$, but it will be more convenient (simplify computations later) to take ``the $(-i)$-supertranspose'' instead.
        
        \item[\cancel] One could choose the abstract version: $\End (U) \times \End(U^*)$ and the superadjoint.
        
        \item[\done] One could also discuss also the $M \times M\sop$ version...
        
        \item[\done] Anyway, $\gl(m|n)$ is isomorphic to $\Skew (R, \vphi)$.
        
        \item The group of automorphisms of $R$ commuting with $\vphi$ is bigger then the group of automorphisms of $M(m,n)$.
        
        \item It is precisely the group of automorphisms of $A$, by \cite{serganova}.
        
        \item[\done] So both Type I and Type II gradings can be recovered this form.
        
        \item[\done] But it is more convenient for Type II gradings: these are the ones that make $R$ simple as a graded algebra, so we can apply the theorem $\End_\D (U)$.
        
        \item[\done] Why is that? There are only 2 ideals on $R$. If one of them is graded, them the other also is (by $1 =  \epsilon_1 + \epsilon_2$). But in this case the top corner is $M(m,n)$ as graded algebra and $\gl(m|n)$ gets its grading by the restriction, hence it is Type I. The converse is true by the same reasoning (and recalling point bellow).
        
        \item[\done] Recall here: If there is an isomorphism between the $\Aut(R, \vphi)$ and $\Aut(S, \psi)$ ($\psi$ could be any map, not necessarily super-anti-auto), them there is a bijection of gradings and of iso classes of gradings.
        
        \item[\done] We will keep the original description for the Type I gradings
        
        \newpage
        
        \item[\done] Subsection: $\vphi$-gradings and bilinear forms
        
        \item[\done] Let $U$ be a *finite dimensional* right $\D$-module. Then $U^*$ is a left $\D$-module or, equivalently, a right $\D\sop$-module.
        
        \item[\done] Note that $U$ is also a *simple* left $R$-module, where $R = \End_\D(U)$. Also note that we can identify $\D = \End_R (U)$ since it is a graded division algebra.
        
        \item[\done] Naturally, $U\Star$ is a right $R$-module. And it must be simple as a $R$-module (proof here).
        
        \item[\done] So we can see $U\Star$ as a simple left $R\sop$-module.
        
        \item[\done] Also note that $\D\sop = \End_{R\sop} (U\Star)$.
        
        \item[\done] Now let $\vphi$ be a (homog of deg e) anti-super-aut on $R$. We can use it to identify $R$ with $R\sop$. In particular, we can consider right $R$-modules as left $R$-modules.
        
        %\item Take $U$ a finite dimensional right $\D$-module. It is naturaly an left $R$-module for $R = \End_\D (U)$. Recall that, by proposition long time ago, that we can then identify $\D = \End_R(U)$ (maps on the right here).
        
        %\item It can be seen as an left $R\sop$-module.
        
        %\item Also, $\D\sop = \End_{R\sop} (U\Star)$.
        
        \item[\done] Using $\vphi$, $U\Star$ is a left $R$-module again (still simple). But we have only one of such up to isomorphism and shift.
        
        \item[\done] Let $\vphi_1$ be such a map. It is not unique, we can compose with a map on the right or a map on the left (which are elements of $\D\sop$ or $\D$, depending on the side.
        
        \item[\cancel] (here the sides of $D$ and $D\sop$ are inverted because of correct composition)
        
        \item[\done] Fixed one $\vphi_1$, we can use it to find an iso between $\D$ and $\D\sop$. Call it $\vphi_0$.
        
        \item We can move $\vphi_1$ to the left using our convention.
        
        \item We can now change $\vphi_1$ to a $\vphi_0$-sesquilinear form $B$.
        
        \item Definition: $\vphi_0$-sesquilinear form.
        
        \item With the pair $(\vphi_0, B)$ we can recover $\vphi$. Two pairs yield the same $\vphi$ if and only if something.
        
        \item start hidden subsection: now suppose $\vphi^2 = \id$.
        
        \item Note that the pair $(\vphi_0\inv, \vphi_0\inv (B (y,x)))$ yields $\vphi\inv$. Hence yields $\vphi$. So there $d\in \D\even_0$ such that...
        
        \item But we are over an algebraically closed field. This implies $d \in \FF$ and $\vphi_0 = \vphi_0\inv$, ie, $\vphi_0$ is a superinvolution.
        
        \item Need here: the formula $\vphi(x) = \Phi\inv x\Star \Phi$.
        
        \newpage
        
        \item Subsection: Odd gradings
        
        \item Again, gradings are odd if $\D$ is odd, so lets see when this can happen.
        
        \item it can't happen for $\D$ simple as algebra! Why? Because them $T$ would have to be an elementary $2$-group. But if $T$ is an elementary $2$-group, them $\vphi_0$ would have order (at least) $4$. (check proof in beamer)
        
        \item If the reader is interested only in $B, C, D, P$, he can stop here.
        
        \item Hidden subsection:
        
        \item example: $\mc O_{\mathrm{\scriptscriptstyle{I\hspace{-.7pt}I}}}$ (basic odd superalgebra of type II)
        
        \item Look bellow for the example
        
        \item First with refinement, by hand
        
        \item Second with loop construction
        
        \item Them show isomorphism
        
        \item Theorem: every odd type II graded division algebra is $\mc O_{II} \tensor \D$, for even $\D$.
        
        \item Next: recall what we are missing from before (eg $V^*$ and $^* V$).
        
        \item Next: Fill what we have till now
        
        \item Next: Isomorphism of $\vphi$-gradings
        
        \item Next: The possible type II gradings on $A$, their relation with $\langle \pi, \tau\rangle$ and isomorphisms.
        
    \end{itemize}

\section{After}


    % \chapter{What do we need before $\vphi$-gradings}
    
    
    
    % \section{Supermodules and superlinear maps}\label{ssec:superlinear-maps}
    
    % Let $R$ be a superalgebra. A \emph{left (right) supermodule} over it (also called \emph{left (right) $R$-supermodule} is simply a left (right) $\ZZ_2$-graded module over it. But we have to be careful in to define the corresponding notion of linear map. It is usual in the theory of modules to write linear maps between left modules on the right and vice-versa. Sometimes, though, it is useful to keep the maps on the same side of the module (when dealing with bimodules, for example). This ``change of side'' is not direct when dealing with supermodules, since one must take the sign rule in consideration.
    
    % \begin{convention}%\label{conv:maps-left-right}
    %     Whenever it is necessary, we will distinguish maps written on the left (called simply as \emph{maps on the left}) from maps writtenon the  right (\emph{maps on the right}) as different concepts.
    % \end{convention}
    
    % \begin{defi}\label{def:side-of-the-map}
    %     Let $R$ be a superalgebra and let $U$ and $V$ be left supermodules over it.
    %     If $\vphi: U \to V$ is a map on the right, we say it is \emph{$R$-superlinear} if
    %     \[
    %         (r\cdot u) \vphi = r\cdot (u)\vphi.
    %     \]
    %     If $\psi: U \to V$ is a map on the left, we say it is \emph{$R$-superlinear} if
    %     \[
    %         \psi(r\cdot u) = \sign{r}{\psi} r\cdot \psi(u).
    %     \]
    %     The definition is analogous for right modules.
    %     By an \emph{isomorphism} of (left or right) $R$-supermodules we mean an invertible $R$-superlinear map (on the left or on the right) which is homogeneous of degree $\bar 0$.
    % \end{defi}
    
    % Unless stated otherwise, we will always consider the $R$-linear maps written on the opposite side of the of the $R$-action on the module, and these are the elements of the sets $\Hom_R (U,V)$ and $\End_R(U)$.
    
    % %Following Definition \ref{def:side-of-the-map}, if $R$ and $S$ are superalgebras and $U$ and $V$ are $(R,S)$-bisupermodulus, then
    
    % There is a natural identification between $R$-superlinear maps on the right and $R$-superlinear maps on the left. In the conditions of the definition above, we can define
    % \[
    %     \vphi^\circ (u) := \sign{\vphi}{u} (u)\vphi
    % \]
    % and
    % \[
    %     (u)^\circ\psi := \sign{\psi}{u} \psi (u).
    % \]
    % The maps $\vphi^\circ$ and $^\circ\psi$ are $R$-superlinear maps but written on the opposite side of $\vphi$ and $\psi$, respectively.
    
    
    % \section{Short version of a $\widehat G$-actions section}
    
    % One of the important tools for dealing with gradings by abelian groups on (super)algebras is the well-known correspondence between  $G$-gradings and $\widehat G$-actions (see e.g. \cite[\S 1.4]{livromicha}), where $\widehat G$ is the algebraic group of characters of $G$, \ie, group homomorphisms $G \rightarrow \FF^{\times}$. The group $\widehat{G}$ acts on any $G$-graded (super)algebra $A = \bigoplus_{g\in G} A_g$ by $\chi \cdot a = \chi(g) a$ for all $a\in A_g$ (extended to arbitrary $a\in A$ by linearity). The map given by the action of a character $\chi \in \widehat{G}$ is an automorphism of $A$. If $\FF$ is algebraically closed and $\Char \FF = 0$, then $A_g = \{ a\in A \mid \chi \cdot a = \chi (g) a\}$, so the grading can be recovered from the action.
    
    % For example, if $A=A\even \oplus A\odd$ is a superalgebra, the action of the nontrivial character of $\ZZ_2$ yields the \emph{parity automorphism} $\upsilon$, which acts as the identity on $A\even$ and as the negative identity on $A\odd$. If $A$ is a $\ZZ$-graded algebra, we get a representation $\widehat \ZZ = \FF^\times \rightarrow \Aut (A)$ given by $\lambda \mapsto \upsilon_\lambda$ where $\upsilon_{\lambda} (x) = \lambda^i x$ for all $x\in A^i$, $i\in \ZZ$.
    
    % A grading on a (super)algebra over an algebraically closed field of characteristic $0$ is said to be \emph{inner} if it corresponds to an action by inner automorphisms. For example, the inner gradings on $\Sl(n)$ (also known as Type I gradings) are precisely the restrictions of gradings on the associative algebra $M_n(\FF)$.
    
    % \section{The Dual of a Superspace}
    
    % We will now discuss the concept of dual of a superspace. If $U = U\even \oplus U\odd$ is a superspace, then $U^*$ is also a superspace with the elementary grading.
    % % If $L: U \to V$ is a homogeneous linear map between superspaces, we define the \emph{superadjoint of $L$} to be the map $L^*: V^* \to U^*$ defined by $L^*(f) = (-1)^{|L||f|} f \circ L$.
    
    % \begin{defi}
    %     Let $L: U \to V$ be linear map between superspaces.
    %     We define the \emph{superadjoint of $L$} to be the map $L^*: V^* \to U^*$ defined by $L^*(f) = (-1)^{|L||f|} f \circ L$ if $L$ is homogeneous and extend it by linearity.
    % \end{defi}
    
    % In the case $U = V$, the map $\vphi: \End(U) \to \End(U^*)$ given by $\vphi(L) = L^*$ is a super-anti-isomorphism of superalgebras, i.e., $\vphi (LS) = (-1)^|L||S| \vphi(S) \vphi(L)$.
    
    % As in the case of finite dimensional vector spaces, we will identify a finite dimensional superspace $U$ with $U^{**}$, but not via the usual evaluation map.
    
    % \begin{defi}
    %     Let $U$ be a superspace. By \emph{superevaluation} we mean any of the two following maps: $ \langle\, ,\rangle : U^* \times U$ defined by $\langle f, u\rangle = f(u)$ and  $ \langle\, ,\rangle : U \times U^*$ defined by $\langle u, f\rangle = (-1)^{|u||f|}f(u)$. 
    % \end{defi}
    
    % With this definition, we have that $\langle T(x), y\rangle = (-1)^{|L||x|}\langle x, L^* (y)\rangle$.
    
    % The map $U \to U^{**}$ given by $u \mapsto  \langle u , \cdot \rangle$ is clearly linear, homogeneous of degree $\bar 0$ and injective. Hence, if $U$ is finite dimensional it is an isomorphism of superspaces, and we will use it to identify $U^{**}$ with $U$.
    
    % \begin{prop}\label{prop:superadjunction-is-involutive}
    %     Let $L: U \to V$ be a linear map between finite dimensional superspaces. Then $L^{**} = L$.
    % \end{prop}
    
    % \begin{proof}
    %     WE REMIND THE AUTHOR TO NOT FORGET IT.
    % \end{proof}
    
    % Differently from the case of usual vector spaces, things do not transfer smoothly to matrix realizations when we fix bases. To be precise, let $\mc B = \{e_1, \ldots, e_k\}$ be a homogeneous basis of a superspace $U$. Given $\lambda \in \FF$, we define the \emph{$\lambda$-superdual basis of $\mc B$} as being the set $\mc B^*_\lambda = \{e_1^*, \ldots, e_k^*\} \subseteq U^*$, where $e_i^*$ is the linear functional defined by $e_i^*(e_j) = \delta_{ij}$ if $e_j$ is even and $e_i^*(e_j) = \lambda \delta_{ij}$ if $e_j$ is odd. In the case $\lambda = 1$, we get the usual dual basis.
    
    % Let $\mc B$ and $\mc C$ be homogeneous bases of the finite dimensional superspaces $U$ and $V$, respectively. Suppose $L: U \to V$ is a linear map whose matrix in relation to the bases above is
    % $\begin{pmatrix}
    %         a & b\\
    %         c & d
    %     \end{pmatrix}$.
    % Then, considering $\mc B^*_\lambda$ and $\mc C^*_\lambda$ on $U^*$ and $V^*$, the matrix of $L^*$ is
    % $\begin{pmatrix}
    %         a\transp & -\lambda c\transp\\
    %         \lambda\inv b\transp & d\transp
    %     \end{pmatrix}$.
        
    % \begin{defi}
    %     Let $X = \begin{pmatrix}
    %         a & b\\
    %         c & d
    %     \end{pmatrix}$ be the a $(n_1 | m_1)\times (n_2 | m_2)$ supermatrix and $\lambda \in \FF$. The \emph{$\lambda$-supertranspose of $X$} is the matrix
    %     \[
    %     \begin{pmatrix}
    %         a\transp & -\lambda c\transp\\
    %         \lambda\inv b\transp & d\transp
    %     \end{pmatrix}. 
    %     \]
    %     %
    %     If $\lambda = 1$, we call it simply the \emph{supertranspose of $X$} and denote it by $X\stransp$.
    % \end{defi}
    
    % Note that the $\lambda$-supertranspose is not involutive in general. This happens because $(\mc B^*_\lambda)^*_\lambda$ and $(\mc B^*_\lambda)^*_\lambda$ do not correspond to the original bases $\mc B$ and $\mc C$ on $U = U^{**}$ and $V=V^{**}$. Actually, they correspond to $\nu (\mc B)$ and $\nu (\mc C)$, respectively, where $\nu$ denotes the parity automorphism.
    
    % \begin{remark}\label{rmk:which-supertranspose-to-use}
    %     Unless stated otherwise and if the original space $U$ is clear from the context, we will use the convention of considering the $1$-superdual basis for passing from $U$ to $U^*$ and the $(-1)$-superdual basis for passing from $U^*$ to $U^{**}$. In particular, if for a fixed homogeneous basis $X$ is the matrix of an operator $L \in \End(U)$, then the supertranspose of $X$ is the matrix of $L^* \in \End(U^*)$; but if $X$ is the matrix of an operator $L\in \End(U^*)$, then the $(-1)$-supertranspose corresponds to $L^* \in \End(U) = \End(U^{**})$.
    % \end{remark}
    
    % Despite of not being involutive, the $\lambda$-supertranspose is an anti-super-auto\-mor\-phism of $M(m,n)$, for every $\lambda \in \FF$. In the case $i = \sqrt{-1}\in \FF$, it may be convenient to use the $(-i)$-supertranspose, since it avoids confusion about where to put the signs: the $(-i)$-supertranspose of $\begin{pmatrix}
    %         a & b\\
    %         c & d
    %     \end{pmatrix}$
    % is $\begin{pmatrix}
    %         a\transp & i c\transp\\
    %         i b\transp & d\transp
    %     \end{pmatrix}$.
        
    % \begin{prop}\label{prop:lambda-supertransp-are-isomorphic}
    %     Let $R = M(m,n)$, $\vphi: R \to R$ be the $1$-supertranspose and $\vphi': R\to R$ be the $\lambda$-supertranspose, for some $\lambda \in \FF$. If $\sqrt{\lambda}\in \FF$, then $(R, \vphi) \iso (R, \vphi')$.
    % \end{prop}
    
    % \begin{proof}
    %     WE REMIND THE AUTHOR TO NOT FORGET
    % \end{proof}
    
    
    % %We define the \emph{first superdual basis} to be the set $\{ e_1^*, \ldots, e_k^* \}$ defined by $e_i^*(e_j) = \delta_{ij}$ (as the usual dual basis).
    
    % % \begin{defi}
    % %     We define the \emph{supertranspose} of a matrix in $M(m,n)$ by
    % %     \[
    % %     \begin{pmatrix}
    % %         a & b\\
    % %         c & d
    % %     \end{pmatrix}\stransp =
    % %     \begin{pmatrix}
    % %         a\transp & -c\transp\\
    % %         b\transp & d\transp
    % %     \end{pmatrix}.
    % %     \]
    % % \end{defi}
    
    % % DEFINITION ONLY FOR SQUARE MATRICES, WE NEED TO CORRECT.
    
    % % Let $U$ and $V$ be finite dimensional superspaces; fix bases for $U$ and $V$ and consider their first superdual bases on $U^*$ and $V^*$. If a matrix $A$ represents a linear map $L: U \to V$, then $A\stransp$ represents $L^*$.
    
    % % The supertranspose, though, is not involutive. This happens because if we start with a homogeneous basis $\mc B$ on $U$ and take first superdual basis twice, we end up with a basis different from $\mc B$ on $U = U^{**}$. WE THEN HAVE THE 2ND SUPERDUAL BASIS.
    
    % % WE HAVE A CONVENTION.
    
    % % If $i = \sqrt{-1} \in \FF$, one could take another approach.
    
    
    % \section{Coarsening by homomorphism and Loop algebra}
    % % -------------------------------------
    
    % \begin{defi}
    % 	Let $G$ and $H$ be groups, $\alpha:G\to H$ be a group homomorphism and $\Gamma:\,A=\bigoplus_{g\in G} A_g$ be a $G$-grading. The \emph{coarsening of $\Gamma$ induced by $\alpha$} is the $H$-grading ${}^\alpha \Gamma: A= \bigoplus_{h\in H} B_h$ where
    % 	$ B_h = \bigoplus_{g\in \alpha\inv (h)} A_g$. (This coarsening is not necessarily proper.)
    % \end{defi}
    
\chapter{Old chapter on $\vphi$-gradings}

The orthosymplectic and the parapletic superalgebras are defined using a nondegenerate bilinenar form. More specifically, if $\vphi$ is the superadjunction related to this form, then we consider the Lie subsuperalgebra of the skew elements with respect to $\vphi$ (see Subsection {\tt ??}). For the remainder of the chapter, $G$ is a fixed abelian group.

We are now going to define the right notion of ``Type I gradings'' for such algebras %(which, as we are going to see, turn out to be all gradings on then).
The problem here is that not all gradings on the associative superalgebra restrict to the Lie superalgebra of skew elements.
%But the restriction actually work if we have a graded superalgebra with superinvolution, \ie, if the the super involtion is homegeneous of degree $e$.

\begin{defi}
    Let $R$ be a superalgebra and let $\vphi: R\to R$ be any map. We call the pair $(R, \vphi)$ a \emph{superalgebra with distinguished map}. If $\vphi$ is a super-anti-automorphism or an superinvolution, we will call it a \emph{superalgebra with super-anti-automorphism} or \emph{superalgebra with superinvolution}, respectively.
\end{defi}

\begin{defi}
    Let $(R, \vphi)$ and $(S, \psi)$ be superalgebras with distinguished map. A \emph{homomorphism} them is a superalgebra homomorphism $f: R\to S$ such that $f\circ \vphi = \psi \circ f$. In particular, the \emph{automorphism group} of $(R, \vphi)$, denoted by $\Aut(R, \vphi)$, consists of all automorphisms of $R$ commuting with $\vphi$.
\end{defi}

We will be more interested in the case $\vphi$ is super-anti-automorphism (or, more specifically, a superinvolution), but the general case is useful to avoid redundancy in the results and to connect regular gradings to $\vphi$-gradings (see Propositions \texttt{?? [bijections between gradings]  and ?? [example of it]}). Note that if the distinguished maps are taken to be the identity, the concepts above reduce to superalgebra and superalgebra homomorphism. 

\begin{defi}
    Let $(R, \vphi)$ be an associative superalgebra with distinguished map. A \emph{$\vphi$-grading} on it is a $G$-grading $\Gamma$ ($G$ is fixed) such that $\vphi$ is homogeneous of degree $e$. In this case, we say that $(R, \vphi)$ equipped with $\Gamma$ is a \emph{graded superalgebra with distinguished map}.
\end{defi}

Again, the concept of $\vphi$-grading reduces to regular gradings when $\vphi = \id$. In the case $\vphi$ is a super-anti-automorphism, the sub-Lie-superalgebra $\Skew (R, \vphi) = \{ r\in R \mid \vphi(r) = -r \}$ is a graded subsuperspace of $R$. This follows from the lemma bellow:

\begin{lemma}
    Let $V$ be a $G$-graded vector space and let $f: V \to V$ be a homogeneous map of degree $e$. Then the eigenspaces of $f$ are graded subspaces.
\end{lemma}

\begin{proof}
    Let $v\in V$ be an eigenvector with eigenvalue $\lambda \in \FF$ and write $v = \sum_{g\in G} v_g$, where $v_g \in V_g$. On one hand, $f(v) = \lambda v = \sum_{g\in G} \lambda v$. On the other hand, $f(v) = \sum_{g\in G} f(v_g)$. Since the decomposition on homogeneous elements is unique, we have that $f(v_g) = \lambda v_g$ for all $g\in G$, completing the proof.
\end{proof}

\section{$\vphi$-gradings and $\widehat G$-actions}

It is important to know how to translate this notion to the language of $\widehat G$-actions.

\begin{prop}
    Let $(R, \vphi)$ be a superalgebra with distinguished map and consider the usual correspondence between gradings on $R$ and $\dual G$-actions. Then gradings on $(R, \vphi)$ correspond to $\dual G$-actions by automorphisms of $(R, \vphi)$.
\end{prop}

\begin{proof}
    Let $\Gamma$ be a grading on $R$. The map $\vphi$ is homogeneous of degree $e$ if, and only if, $\vphi$ being $\widehat G$-equivariant (Proposition \ref{prop:homogeneous-deg-e-widehatG}), which means $\chi\cdot \vphi (r) = \vphi (\chi\cdot r)$ for all $\chi \in \widehat G$ and all $r \in R$. But this is equivalent to say that $\eta_\Gamma(\chi)$ commutes with $\vphi$ for every $\chi \in \dual G$ or, in other words, that $\eta_\Gamma(\chi) \in \Aut (R, \vphi)$. 
\end{proof}

\begin{prop}\label{prop:homogeneous-deg-e-widehatG}
    Let $V$ and $W$ be graded vector spaces. A linear map $f: V \to W$ is homogeneous of degree $e$ if, and only if, $f$ is a $\widehat G$-equivariant map.
\end{prop}

\section{Relation to gradings on superalgebras of type $A$}

% As we have seem (Section {\tt ??}), all gradings on $M(m+1, n+1)$ restrict to gradings on $A(m, n)$, given us the Type I gradings. Our strategy to obtain the Type II gradings is to embed $\gl(m+1 | n+1)$ in a different associative superalgebra, such that these gradings are restrictions of gradings on the associative superalgebra.

Our goal here is to construct a different model of $\gl(m+1 | n+1)$. More precisely, we want to embed $\gl(m+1 | n+1)$ in an associative superalgebra $\mc R$ such that every Type II grading on $A(m,n)$ is induced from some grading on $\mc R$.

Let $R = M(m+1, n+1)$ and let $\psi: R\to R$ be any super-anti-automorphism. Then consider the superalgebra $\mc R = R\times R$. The map $\vphi: \mc R \to \mc R$ given by $\vphi(x,y) = (\psi\inv(y), \psi(x))$ is a superinvolution on $\mc R$. Then the Lie superalgebra $\Skew (\mc R, \vphi) = \{(r, -\psi(r) | r\in R\}$ is isomorphic to $R^{(-)} = \gl(m+1, n+1)$ by the projection on the first component $\mc R = R\times R \to R$.

\begin{remark}
    For any superalgebra $R$, we could consider the superalgebra $\mc R = R \times R\sop$ and the superinvolution $\vphi: \mc R \to \mc R$ defined by $\vphi(x, y) = (y, x)$. The construction above reduces to this one if we identify $R\sop$ with $R$ via the super-anti-automorphism $\psi$. 
\end{remark}

We can choose any super-anti-isomorphism for $\psi$. One choice would be the supertransposition, but we are going to use a different one that will simplify some computations later. Let $\psi: R \to R$ be the map given by:
\[
    \psi\left(
    \begin{pmatrix}
        a & b\\
        c & d
    \end{pmatrix}\right) = 
    \begin{pmatrix}
        a\transp & i c\transp\\
        i b\transp & d\transp
    \end{pmatrix},
\]
where $i$ is a square root of $-1$.

\begin{remark}
    If we take the basis $\{f_1, \ldots, f_k\}$ on $U^*$ given by $f_j = e_j $ if $|j| = \barr 0$ and $f_j = ie_j$ if $|j| = \barr 1$, then the matrix $\psi (M)$ represents the superadjoint of the operator represented by $M$. Also, $(R, \psi)$ is isomorphic to $R$ with the supertransposition.
\end{remark}

\begin{prop}
    Gradings on $A(m,n)$ are in bijection with $\vphi$-gradings on $\mc R$.
\end{prop}

\begin{proof}
    {\tt needed}
\end{proof}

As we see, not only the Type II gradings on $A(m,n)$ can be recovered this way, but also the Type I gradings. But we will continue to use our previous model to describe the Type I gradings. This is because Type I gradings do not come from $\vphi$-gradings on $\mc R$ making it simple as a graded algebra (Proposition \ref{prop:typeII-iff-RxR-simple}, bellow), so we cannot use Theorem {\tt ??}. The situation is the opposite for Type II gradings.

\begin{prop}\label{prop:typeII-iff-RxR-simple}
    A $\vphi$-grading on $\mc R$ makes it simple as a graded algebra if, and only if, it restricts to a Type II grading.
\end{prop}

\begin{proof}
    Note that $\mc R$ has only two nontrivial ideals, $I = R\times 0$ and $J = 0\times R$.
    
    Let $\Gamma$ be a grading on $\mc R$ and suppose $I$ is a graded ideal. Then $I$ is a graded superalgebra and, hence, $\epsilon_1 := (1,0)$ is homogeneous of degree $e$. Since $(1,1)$ is also homogeneous of degree $e$, so it is $\epsilon_2:= (0,1)$. Then $J = \epsilon_2 \mc R$ is also a graded ideal. By symmetry, we get that $(\mc R, \Gamma)$ is not graded simple if, and only if, $\epsilon_1$ is homogeneous (of degree $e$).
    
    Note that if $\epsilon_1$ is homogeneous of degree $e$, then the projection $\mc R = R\times R \to I$ is homogeneous of degree $e$ and its restriction to $\Skew(\mc R, \vphi)$ is an isomorphism, which implies that the grading on $\gl(m +1, n+1)$ can be obtained by a restriction of a grading on the associative superalgebra $I = R\times 0$.
    
    On the other hand, given a Type I grading on $A(m,n)$, it comes from a grading on $R$. If we consider the grading on $\mc R = R\times R$ by taking the direct sum of graded algebras {\tt (Definition ??)}, then $\mc R$ is not a simple graded algebra.
\end{proof}

\section{Temporary section for references}

Duals of graded supermodules are introduced for the proof of isomorphism theorem of Type I gradings on $A$.

\begin{convention}\label{conv:maps-left-right}
    Whenever it is necessary, we will distinguish maps written on the left (called simply as \emph{maps on the left}) from maps written on the  right (\emph{maps on the right}) as different concepts.
    
    If $\U$ is a left (right) $\mc R$-supermodule, we use the convention of writing the $\mc R$-linear maps on the right (left).
\end{convention}



The following is the converse of the Graded Density Theorem.

\begin{prop}\label{prop:converse-density-thm}
    Let $\D$ be a graded division superalgebra and let $\U$ be a finite dimensional graded right $\D$-supermodule. Consider $\mc R := \End_\D (\U)$. Then $\U$ is simple as a left $\mc R$-supermodule and $\D \iso \End_{\mc R} (\U)$ via the $\D$-action.
    (we will use this isomorphism to identify $\D$ with $\End_{\mc R} (\U)$). 
\end{prop}

\begin{proof}
Let $u \in \U$ be a nonzero vector and complete it to a basis $\{u\} \cup \{ u_\lambda\}_{\lambda \in \Lambda}$, where $\Lambda$ is an indexing set.

Let $v\in \U$ be any element. We can then define $r\in \End_\D (\U)$ by $r(u) = v$ and $f(u_\lambda) = 0$ for all $\lambda \in \Lambda$. Since $u \neq 0$ was arbitrary, this shows $\U$ is a simple $\mc R$-module. 

The statement that the $\mc R$ action is $\D$-linear is, clearly, the same of the $\D$-action being $\mc R$-linear: $r(vd) = (rv)d$ for all $r\in \mc R$ and $d \in \D$. So the $\D$-action give us, indeed, a map $\D \to \End_{\mc R} (\U)$.
Since $\D$ is graded simple and its action is nonzero ($v\cdot 1 = v$ for all $v \in V$), this map is injective.

To show it is surjective, consider $r$ as defined above, but with $v = u$. Let $f\in \End_{\mc R} (U)$. Since $\U$ is a simple $\mc R$-supermodule, $f$ is determined by its value on $u$.
We have that $uf = u d + \sum_{\lambda \in \Lambda} u_\lambda d_\lambda$ for $d, d_\lambda \in \D$ (where all but finitely many $d_\lambda = 0$). 
Them $ uf = (ru)f = r(uf) = r(ud + \sum_{\lambda \in \Lambda} u_\lambda d_\lambda )  = ud$. Hence $f = d$, concluding the proof.
\end{proof}

% ---------------------------------------------------
\section{Superdual of a graded module}\label{ssec:superdual}
% ---------------------------------------------------

We will need the following concepts. Let $\D$ be an associative superalgebra with a grading by an abelian group $G$, so we may consider $\D$ graded by the group $G^\# = G\times \ZZ_2$. Let $\U$ be a $G^\#$-graded \emph{right} $\D$-module. The parity $|x|$ of a homogeneous element $x \in \D$ or $x\in \U$ is determined by $\deg x \in G^\#$. The \emph{superdual module} of $\U$ is $\U\Star = \Hom_\D (\U,\D)$, with its natural $G^\#$-grading and the $\D$-action defined on the \emph{left}: if $d \in \D$ and $f \in \U \Star$, then $(df)(u) = d\, f(u)$ for all $u\in \mc U$.

We define the \emph{opposite superalgebra} of $\D$, denoted by $\D\sop$, to be the same graded superspace $\D$, but with a new product $a*b = (-1)^{|a||b|} ba$ for every pair of $\ZZ_2$-homogeneous elements $a,b \in \D$. The left $\D$-module $\U\Star$ can be considered as a right $\D\sop$-module by means of the action defined by $f\cdot d := (-1)^{|d||f|} df$, for every $\ZZ_2$-homogeneous $d\in \D$ and $f\in \U\Star$.

\begin{lemma}\label{lemma:Dsop}
	If $\D$ is a graded division superalgebra associated to the pair $(T,\beta)$, then $\D\sop$ is associated to the pair $(T,\beta\inv)$.\qed
\end{lemma}

If $\U$ has a homogeneous $\D$-basis $\mc B = \{e_1, \ldots, e_k\}$, we can consider its \emph{superdual basis} $\mc B\Star = \{e_1\Star, \ldots, e_k\Star\}$ in $\U\Star$, where $e_i\Star : \U \rightarrow \D$ is defined by $e_i\Star (e_j) = (-1)^{|e_i||e_j|} \delta_{ij}$.

\begin{remark}\label{rmk:gamma-inv}
	The superdual basis is a homogeneous basis of $\U\Star$, with $\deg e_i\Star = (\deg e_i)\inv$. So, if $\gamma = (g_1, \ldots, g_k)$ is the $k$-tuple of degrees of $\mc B$, then $\gamma\inv = (g_1\inv, \ldots, g_k\inv)$ is the $k$-tuple of degrees of $\mc B\Star$.
\end{remark}

For graded right $\D$-modules $\U$ and $\V$, we consider $\U\Star$ and $\V\Star$ as right $\D\sop$-modules as defined above. If $L:\U \rightarrow \V$ is a $\ZZ_2$-homogeneous $\D$-linear map, then the \emph{superadjoint} of $L$ is the $\D\sop$-linear map $L\Star: \V\Star \rightarrow \U\Star$ defined by $L\Star (f) = (-1)^{|L||f|} f \circ L$. We extend the definition of superadjoint to any map in $\Hom_\D (\U, \V)$ by linearity.

\begin{remark}
	In the case $\D=\FF$, if we denote by $[L]$ the matrix of $L$ with respect to the homogeneous bases $\mc B$ of $\U$ and $\mc C$ of $\V$, then the supertranspose $[L]\sT$ is the matrix corresponding to $L\Star$ with respect to the superdual bases $\mc C\Star$ and $\mc B\Star$.
\end{remark}

We denote by $\vphi: \End_\D (\U) \rightarrow \End_{\D\sop} (\U\Star)$ the map $L \mapsto L\Star$. It is clearly a degree-preserving super-anti-isomorphism. It follows that, if we consider the Lie superalgebras $\End_\D (U)^{(-)}$ and $\End_{\D\sop} (U\Star)^{(-)}$, the map $-\vphi$ is an isomorphism.

We summarize these considerations in the following result:

\begin{lemma}\label{lemma:iso-inv}
	If $\Gamma = \Gamma(T,\beta,\gamma)$ and $\Gamma' = \Gamma(T,\beta\inv,\gamma\inv)$ are $G$-gradings (considered as $G^\#$-gradings) on the associative superalgebra $M(m,n)$, then, as gradings on the Lie superalgebra $M(m,n)^{(-)}$, $\Gamma$ and $\Gamma'$ are isomorphic via an automorphism of $M(m,n)^{(-)}$ that is the negative of a super-anti-automorphism of 	$M(m,n)$.
\end{lemma}

\begin{proof}
	Let $\D$ be a graded division superalgebra associated to $(T,\beta)$ and let $\U$ be the graded right $\D$-module associated to $\gamma$. The grading $\Gamma$ is obtained by an identification $\psi: M(m, n) \xrightarrow{\sim} \End_\D (\U)$. By Lemma \ref{lemma:Dsop} and Remark \ref{rmk:gamma-inv}, $\Gamma'$ is obtained by an identification $\psi': M(m, n) \xrightarrow{\sim} \End_{\D\sop} (\U\Star)$. Hence we have the diagram:

	\begin{center}
		\begin{tikzcd}
			& \End_\D (\U) \arrow[to=3-2, "-\vphi"]\\
			M(m, n) \arrow[ur, "\psi"] \arrow[dr, "\psi'"]\\
			& \End_{\D\sop} (\U\Star)
		\end{tikzcd}
	\end{center}

	Thus, the composition $(\psi')\inv \, (-\vphi) \, \psi$ is an automorphism of the Lie superalgebra $M(m,n)^{(-)}$ sending $\Gamma$ to $\Gamma'$.
\end{proof}


\chapter{Gradings on Associative Superalgebras with Superivolution}

The orthosymplectic and the parapletic superalgebras are defined using a nondegenerate bilinenar form. 
More specifically, if $\vphi$ is the superadjunction related to this form, then we consider the Lie subsuperalgebra of the skew elements with respect to $\vphi$ (see Subsection {\tt ??}). 

\vspace{5mm}
{\tt (More intro to come...)}

% For the remainder of the chapter, $G$ is a fixed abelian group.

% In this chapter we are going to describe 

\section{Super-anti-automorphisms and sesquilinear forms}

Let $\D$ be a graded division superalgebra and let $\U$ be a finite dimensional graded right $\D$-supermodule over it. Consider the graded superalgebra $R := \End_\D(\mc U)$. Then $\mc U$ is a $(R, \D)$-bisupermodule. By Proposition \ref{prop:converse-density-thm}, $\mc U$ is a simple graded left supermodule over $R$ and we have a natural identification between $\D$ and $\End_R(\mc U)$.

Recall {\tt (Section ??)} that $\mc U\Star = \Hom_\D (\mc U, \D)$ is a graded left $\D$-supermodule or, equivalently, a graded right $\D\sop$-supermodule. Analogously, we can make $\mc U\Star$ a graded right $R$-supermodule by defining $(f\cdot r) (u) := f(r\cdot u)$ for all $f\in \mc U\Star$, $r\in R$ and $u\in \mc U$ (i.e., $f\cdot r = f \circ r$, since $R = \End_\D(\mc U)$). Hence, we can consider $\mc U\Star$ as a graded left $R\sop$-supermodule via $r\cdot f = (-1)^{|r||f|} f\cdot r$ % (i.e., $r\cdot f = r\Star (f)$)
and $\mc U\Star$ becomes a $(R\sop, \D\sop)$-bisupermodule.

\begin{lemma}\label{lemma:U-star-R-sop}
    The left $R\sop$-supermodule $\mc U\Star$ is a graded simple and the action of $\D\sop$ give us an isomorphism $\D\sop \to \End_{R\sop}(\mc U\Star)$.
\end{lemma}

\begin{proof}
    By definition, the $R\sop$-action correspond to the representation $\rho: R\sop \to \End_{\D\sop} (\mc U\Star)$ given by $ \rho(r) = r\Star$, which is an isomorphism since $\U$ is finite dimensional as a $\D$-supermodule. If we use $\rho$ to identify $R\sop$ with $\End_{\D\sop} (\mc U\Star)$, the result follows from Proposition \ref{prop:converse-density-thm}.
\end{proof}

%Note that the $R\sop$-action correspond to the representation $\rho: R\sop \to \End_{\D\sop} (\mc U\Star)$ given by $ \rho(r) = r\Star$, which is an isomorphism. If we use it to identify $R\sop$ with $\End_{\D\sop} (\mc U\Star)$, we can then apply Proposition \ref{prop:converse-density-thm} again and conclude that $\mc U\Star$ is a graded simple left $R\sop$-module and that we naturally can identify $\D\sop$ with $\End_{R\sop}(\mc U\Star)$.

Now let $\vphi: R\to R$ be a super-anti-automorphism homogeneous of degree $e$.
We can see it as an isomorphism $\vphi: R\to R\sop$ and use it to identify $R$ with $R\sop$. In particular, we will make the left $R\sop$-action into an left $R$-action via $r\cdot f = \vphi(r)f$, for all $r\in R$ and all $f\in \U\Star$. 
Note that this is the same as defining the left $R$-action using the right $R$-action by
%
\begin{equation}\label{eq:R-action-back-on-the-right}
    r\cdot f = \sign{r}{f} f \circ \vphi(r) .
\end{equation}

We will now consider $\U\Star$ as an $(R,\D\sop)$-superbimodule. 
In particular, the superalgebra $\End_R(\U\Star)$ should be understood as the set of $R$-linear maps with respect to the $R$-action on the left. 
Also, since the action on the left, we will follow the convention of writing $R$-linear maps on the right.

From Lemma \ref{lemma:U-star-R-sop}, it follows that $\U\Star$ is a simple graded left $R$-supermodule and that we can identify $\D\sop$ with $\End_R (\U\Star)$. 
But from Proposition {\tt ??}, $R$ has only one graded simple supermodule up to isomorphism and shift, hence there is an invertible $R$-linear map $\vphi_1: \mc U \to \mc U\Star$ which is homogeneous of some degree $(g_0, \alpha)\in G^\#$.

Using the identifications $\D = \End_R(\U)$ and $\D\sop = \End_R(\U\Star)$ introduced above, we can define $\vphi_0: \D \to \D\sop$ by $\vphi_0(d) := \sign{\vphi_1}{d}\vphi_1\inv d \vphi_1$ (where juxtaposition denotes composition of maps on the right). It is straightforward to check that $\vphi_0$ is an isomorphism and, hence, we can consider it as a super-anti-automorphism of $\D$.

Note that
\begin{equation}\label{eq:sesquilinear-before-B}
    (ud)\vphi_1 = u (d\vphi_1) =  u(\vphi_1\vphi_1\inv d \vphi_1) = \sign{d}{\vphi_1}(u\,\vphi_1 )\vphi_0(d).
\end{equation}

\begin{defi}
    Let $V$ and $V'$ be left $R$-supermodules and let $\psi: V \to V'$ be a $R$-linear map. We define $\psi^\circ: V \to  V'$ to be the following map, written on the left:
    \[
        \psi^\circ(v) = \sign{\psi}{v} (v)\psi.
    \]
\end{defi}

\begin{lemma}
    Let $\psi: V \to V$ and $\theta: V' \to V''$ be $R$-linear maps between left $R$-supermodules. Then, for all $r\in R$ and $v\in V$,
    \begin{enumerate}[(i)]
        \item $\psi^\circ (rv) = \sign{\psi}{r} r\psi(v)$;
        \item $(\psi\theta) = \sign{\psi}{\theta} \theta^\circ \psi^\circ$.
    \end{enumerate}
    
\end{lemma}

\begin{remark}
    The map $\psi^\circ$ defined above is not, in general, $R$-linear. Instead, it satisfies $\psi^\circ (rv) = \sign{\psi}{v} r \psi^\circ (v)$, for all $r\in R$ and $v\in V$. 
    Also, given another $R$-linear map $\theta: \V' \to \V''$, we have that $(\psi\theta)^\circ = \sign{\psi}{\theta} \theta^\circ\psi^\circ$.


    % Let $\psi: \V \to \V'$ and $\theta: \V' \to \V''$ be $R$-linear maps between left $R$-supermodules. Then
    % \begin{enumerate}[(i)]
    %     \item $\psi^\circ (rv) = \sign{\psi}{v} r \psi^\circ (v)$, for all $r\in R$ and $v\in V$;
    %     \item $(\psi\theta)^\circ = \sign{\psi}{\theta} \theta^\circ\psi^\circ$.
    % \end{enumerate}
\end{remark}

Using the notation just introduced and Equation \eqref{eq:sesquilinear-before-B}, we have that
%
\begin{equation*}%\label{eq:sesquilinear-with-vphi1-circ}
    \begin{split}
        \vphi_1^\circ (ud) &= (-1)^{|\vphi_1|(|u|+|d|)} (ud)\vphi_1 \\
        &=\sign{\vphi_1}{u}(u\,\vphi_1 )\vphi_0(d) = 
        \vphi_1^\circ (u) \vphi_0 (d)
    \end{split}
\end{equation*}
%
Recalling the definition of the right $\D\sop$-action on $\U\Star$, this can be rewritten using the left $\D$-action as
%
\begin{equation}\label{eq:sesquilinear-with-vphi1-circ}
    \vphi_1^\circ (ud) = (-1)^{|d|(|\vphi_1|+|u|)}\vphi_0 (d)\vphi_1^\circ (u).
\end{equation}


% We define $\vphi_1 := \tilde\vphi_1^\circ$ and $\vphi_0: \D \to \D$ by $\vphi_0(d): \sign{\vphi_1}{d} \vphi_1 d^\circ \vphi_1\inv$. 
% Note that the left $\D$-action on $\U\Star$ is connected the right $\D\sop$-action by the formula $f\cdot d = d^\circ \cdot f$, hence $\vphi_0$ is an anti-super-automorphism of $\D$. Also, we have that $\vphi_1(ud) = \sign{u}{d} \vphi_1(d^\circ  $

We define $B: \U\times \U \to \D$ to be the $\FF$-bilinear map given by $B(u, v) = \vphi_1^\circ (u)(v)$. 
Considered as a $\FF$-linear map $B: \U\tensor \U \to \D$, it is homogeneous of the same parity and degree of $\vphi_1$.

We will now see that, considering ``scalars'' on $\D$, the map $B$ is \emph{$\vphi_0$-sesquilinear}, \ie, for all $u,v \in \U$ and $d\in \D$, we have that
\begin{equation}\label{eq:easy-one}
    B(u,vd) = B(u,v)d
\end{equation}
and
\begin{equation}\label{eq:hard-one}
    B(ud, v) = (-1)^ {(|B| + |u|)|d|}\vphi_0(d) B(u, v).
\end{equation}

Since $\vphi_1^\circ (u) \in \U\Star$, we have that $\vphi_1^\circ(u)(vd) = \vphi_1^\circ(u)(v)d$, which give us \eqref{eq:easy-one}.
Equation \ref{eq:hard-one} is a restatement of \eqref{eq:sesquilinear-with-vphi1-circ}.

Finally, Equation \eqref{eq:R-action-back-on-the-right} give us
%
\begin{equation*}
    \begin{split}
        B(ru,v) &= \vphi_1^\circ (ru)(v) = \sign{r}{\vphi_1} r \vphi_1^\circ (u)(v)\\ &= \sign{r}{u} \big(\vphi_1^\circ (u) \circ \vphi(r) \big) = \sign{r}{u} B(u,\vphi(r)v),
    \end{split}
\end{equation*}

\begin{prop}
    
\end{prop}

% it is clear that $B(u,vd) = B(u,v)d$. By Equation \eqref{eq:sesquilinear-before-B}, the map $B: \U \times \U \to \D$ is a \emph{$\vphi_0$-sesquilinear form}, \ie,
% \[
%     B(ud, v) = (-1)^ {(|B| + |u|)|d|}
% \]

% Consider the map $\vphi_0: \D \to \D$ given by 

% Consider the map $\vphi_0(d) := \sign{\vphi_1}{d}\vphi_1\inv d \vphi_1$ (where juxtaposition denotes composition of maps on the right).
% Then
% \begin{equation}\label{eq:sesquilinear-before-B}
%     (ud)\vphi_1 = (u) (d\vphi_1) =  (u)(\vphi_1\vphi_1\inv d \vphi_1) = \sign{d}{\vphi_1}(u)\vphi_1 \,\vphi_0(d).
% \end{equation}
% It is straightforward to check that $\vphi_0: \D = \End_R(\mc U) \to \End_R(\mc U\Star) = \D\sop$ is an isomorphism of superalgebras.
% We will consider $\vphi_0$ as a super-anti-automorphism on $\D$.

%Recall (Subsection \ref{ssec:superlinear-maps}) that $\vphi_1^\circ$ is an $R$-superlinear map on the left. Using it, the Equation \eqref{eq:sesquilinear-before-B} becomes
% \begin{equation}\label{eq:sesquilinear-with-vphi1-circ}
%     \vphi_1^\circ (ud) = \vphi_1^\circ (u) \vphi_0 (d)
% \end{equation}

%For each $u\in \U$, $\vphi_1(u)$ is an $\D$-linear map on the left. We, then, can define \[B(u,v) = \sign{\vphi_1}{u} ((u)\vphi_1)(v),\]
%or, using the notation introduced in Subsection \ref{ssec:superlinear-maps},
% \[
%     B(u,v) = \vphi_1^\circ (u) (v),
% \]
% for all $u,v \in U$.


%We will consider $\vphi_0$ as a super-anti-automorphism on $\D$.

%Recall (Subsection \ref{ssec:superlinear-maps}) that $\vphi_1^\circ$ is an $R$-superlinear map on the left. Using it, the Equation \eqref{eq:sesquilinear-before-B} becomes


%Since  this new action makes $\mc U\Star$ a simple graded left $R$-supermodule. Also, since we had $\D\sop = \End_{R\sop} (\U\Star)$, we c it follows that $\D\sop$ can be identif
%We, then, have that $\mc U\Star$ is a simple graded left $R$-module and that $\D\sop = \End_R(\mc U\Star)$.

%Since $R$ acts on the left, we will use the convention of writing $R$-superlinear maps on the right. % (see Subsection \ref{ssec:superlinear-maps}).
%By Proposition {\tt ??}, $R$ has only one graded simple supermodule up to isomorphism and shift, \ie, there is an invertible $R$-linear map $\vphi_1: \mc U \to \mc U\Star$ which is homogeneous of some degree $(g_0, \alpha)\in G^\#$.



% Using the notation just introduced, Equation \eqref{eq:sesquilinear-before-B} becomes
% \begin{equation}%\label{eq:sesquilinear-with-vphi1-circ}
%     \vphi_1^\circ (ud) = \vphi_1^\circ (u) \tilde\vphi_0 (d)
% \end{equation}
% or, puting $\vphi_0(d) = (\tilde \vphi_0(d))^\circ$,
% \begin{equation}\label{eq:sesquilinear-with-vphi1-circ}
%     \vphi_1^\circ (ud) = (-1)^{|d|(|\vphi_1|+|u|)}\vphi_0 (d)\vphi_1^\circ (u).
% \end{equation}







It is important to note that the map $\vphi_1$ is not uniquely determined. For every nonzero homogeneous element $d\in\D$, the map $d\vphi_1: \mc U \to \mc U\Star$ is also an invertible, homogeneous and $R$-superlinear. If we consider $\vphi_1$ fixed, every such map can be obtained this way: if $\vphi_1' : \mc U \to \mc U\Star$ is an invertible homogeneous $R$-superlinear map, then we could take $d := \vphi'\vphi\inv \in \End_R(\mc U) = \D$ and, hence, $\vphi_1' = d \vphi_1$.

Changing $\vphi_1$ to $\vphi_1'$ also changes the super-anti-automorphism $\vphi_0 (c) = \sign{\vphi_1}{d}\vphi_1\inv d \vphi_1$ to $\vphi_0':= \sign{\delta\vphi_1}{d}\vphi_1\inv\delta\inv d \delta\vphi_1= \sign{\delta}{d} \vphi_0(\delta\inv d \delta)$. In other words, $\vphi_0' = \vphi_0 \circ \operatorname{sInt}_\delta$, where $\operatorname{sInt}_\delta: $

The choice of the map $\vphi_1$ 

%If $\vphi_1' : \mc U \to \mc U\Star$ is another invertible and homogeneous $R$-superlinear map, then $\vphi'\vphi\inv: \mc U \to \mc U$ is a nonzero homogeneous element of $\D = \End_R(\mc U)$.
%Conversely, for each nonzero homogeneous element $\delta\in\D$, the map $\vphi_1' := \delta\vphi_1: \mc U \to \mc U\Star$ is $R$-superlinear, invertible and homogeneous.
Changing $\vphi_1$ to $\vphi_1'$ also changes the super-anti-automorphism $\vphi_0(d) = \sign{\vphi_1}{d}\vphi_1\inv d \vphi_1$ to $\vphi_0':= $


depends on the choice of $\vphi_1$: if we consider $\delta\vphi_1$ instead, then we change $\vphi_0(d) = \vphi_1\inv d \vphi_1$ to $\vphi_0'(d) = \vphi_0(\delta\inv d \delta)$.

\section{Graded division superalgebras with super-anti-automorphism}
% -----------------------------------------


Let $\D$ be a $G$-graded division superalgebra, associated to the pair $(T, \beta)$ and with parity map $| \cdot |: T \to \ZZ_2$.

Since each component of $\D$ is one-dimensional, an invertible homogeneous map $\vphi: \D \to \D$ is completely determined by a map $\eta: T \to \FF^\times$ such that $\vphi(X_t) = \eta(t) X_t$ for all $t\in T$ and $X_t\in \D_t$.
The map $\vphi$ is a super-anti-automorphism if, and only if,
%
\begin{align*}
    &\vphi(X_a X_b) = (-1)^{|a||b|} \vphi(X_b) \vphi(X_a)\\ \iff& \eta(ab)X_a X_b = (-1)^{|a||b|} \eta(a) \eta(b) X_b X_a\\ \iff& \eta(ab)X_a X_b = (-1)^{|a||b|} \eta(a) \eta(b) \beta(b,a) X_a X_b\\ \iff& \eta(ab) = (-1)^{|a||b|} \eta(a) \eta(b) \beta(b,a). \numberthis \label{eq:eta-super-anti-auto}
\end{align*}

\begin{prop}
    If $\D$ admits a super-anti-automorphism, then $\beta$ can only take values $\pm 1$.
    Moreover, if $\D$ is simple as an algebra, then $T$ is an elementary $2$-group.
\end{prop}

\begin{proof}
    Let $X_a\in \D_a$ and $X_b\in \D_b$, $a,b\in T$, be nonzero.
    By definition of $\beta$, $X_a X_b = \beta(a,b) X_b X_a$.
    Applying a super-anti-automorphism $\vphi$, we get \[(-1)^{|a||b|} \eta(a) \eta(b) X_b X_a = \beta(a,b) (-1)^{|a||b|} \eta(a) \eta(b) X_a X_b,\] hence $X_b X_a = \beta(a,b) X_a X_b = \beta(a,b)^2 X_b X_a$.
    It follows that $\beta(a,b)^2 = 1$, proving the first assertion.
    
    We now assume $\D$ simple as algebra, i.e., $\beta$ nondegenerate.
    If $a\in T$ has order $n>2$, then $\beta(a, \cdot)\in \widehat T$ would had order $n$.
    But $\beta(a, \cdot )^2 =1$, as we just saw.
\end{proof}

\begin{cor}\label{cor:super-anti-order-4}
    Suppose $\D$ is simple as an algebra.
    If $\D\odd = 0$, every super-anti-automorphism is a superinvolution, but if $\D\odd \neq 0$, every super-anti-automorphism has order $4$.
\end{cor}

\begin{proof}
    Let $a\in T$ and $0 \neq X_a \in \D_a$. If $a$ is even, Equation \eqref{eq:eta-super-anti-auto} implies $\eta(e) = \eta (a^2) = \eta(a)^2$, hence $\eta(e) = 1$ (case $a=e$) and, then,  $\eta(a) \in \{ \pm 1 \}$. If $\D\odd = 0$, this means $\vphi^2 = \id$. Now, if $a$ is odd, Equation \eqref{eq:eta-super-anti-auto} implies $1 = \eta(e) = \eta (a^2) = - \eta(a)^2$, hence $\eta(a) \in \{ \pm i \}$, where $i = \sqrt{-1}$. This implies $\vphi$ has order $4$.
\end{proof}

As a consequence of Corollary \ref{cor:super-anti-order-4}, all Type I gradings on the superalgebras in the series $B$, $C$, $D$ and $P$ are even.

We will now focus on the case $\D$ is odd, i.e., $\D\odd \neq 0$.

\begin{ex}\label{ex:supertransp-graded}
    Consider on $M(1,1)$ the $\ZZ_2$-grading given by declaring $\begin{pmatrix}
       1 & 0 \\
       0 & 1
     \end{pmatrix}$ and
     $\begin{pmatrix}
       0 & i \\
       1 & 0
     \end{pmatrix}$ to have degree $\barr 0$ and
     $\begin{pmatrix}
       -1 & 0 \\
       0 & 1
     \end{pmatrix}$ and
     $\begin{pmatrix}
       0 & -i \\
       1 & 0
     \end{pmatrix}$ to have degree $\barr 1$. In other words, as a $\ZZ_2^\#$-grading on $M_2(\FF)$ we have\\
     %
     \begin{center}
     \begin{tabular}{ l c r }
     $\deg \begin{pmatrix}
      \phantom{-}1 & 0\phantom{..} \\
      \phantom{-}0 & 1\phantom{..}
     \end{pmatrix} = (\bar 0, \bar 0)$, && $\deg \begin{pmatrix}
      \phantom{.}0 & \phantom{-}i\phantom{.} \\
      \phantom{.}1 & \phantom{-}0\phantom{.}
     \end{pmatrix} = (\bar 0, \bar 1)$,\\
     $\deg \begin{pmatrix}
      -1 & 0\phantom{..} \\
       \phantom{-}0 & 1\phantom{..}
     \end{pmatrix} = (\bar 1, \bar 0)$ &
     and
     & $\deg \begin{pmatrix}
      \phantom{.}0 & -i\phantom{.} \\
      \phantom{.}1 & \phantom{-}0\phantom{.}
     \end{pmatrix} = (\bar 1, \bar 1)$,
     \end{tabular}
     \end{center}
     %
     which is clearly a division grading. The supertranspose is a super-anti-automorphism on it.
\end{ex}

\begin{ex}\label{ex:FZZ_4-with-superinvoltion}
    Let $T= \langle \omega \rangle$ where $\omega$ has order $4$, $\beta$ be trivial and $| \cdot |: T \to \ZZ_2$ be the only nontrivial homomorphism. In other words, $\D = \FF T$ with $\D\even = \FF 1 \oplus F \omega^2$ and $\D\odd = \FF \omega \oplus \omega^3$. We define a superivolution by $\eta (1) = 1$, $\eta (\omega) = 1$, $\eta (\omega^2) = -1$ and $\eta(\omega^3) = -1$. %We will denote this superinvolution by 
    
    One could verify by hand that $\eta$ satisfies Equation \eqref{eq:eta-super-anti-auto}, but it also follows from the following argument. In both $\D$ and $\D\sop$, $\omega$ is an element with order $4$ and both are superalgebras with dimension $4$. Hence, both homomorphisms $\theta: \frac{\FF[x]}{\langle x^4 - 1 \rangle} \to \D$ and $\theta': \frac{\FF[x]}{\langle x^4 - 1 \rangle} \to \D\sop$ sending $x$ to $\omega$ are isomorphisms (note that $\theta(x^2) = \omega^2$ and $\theta'(x^2) = (\omega\sop)^2 = - \omega^2$, so they are different maps!). The map $\theta'\circ \theta\inv: \D \to \D\sop$ corresponds to an super-anti-automorphism with the $\eta$ above.
\end{ex}

Example \ref{ex:FZZ_4-with-superinvoltion} is our first example of odd graded division superalgebra with superinvolution. We can use it to transform any even graded division superalgebra into an odd one by ``extending scalars''.

\begin{lemma}
    Let $(R, \vphi)$ and $(S, \psi)$ be graded superalgebras with superinvolution, with grading groups $G$ and $H$, respectively. Then $(R\tensor S, \vphi\tensor \psi)$ is a $G\times H$-graded superalgebra with superinvolution.\qed
\end{lemma}

\begin{lemma}
    Let $\D$ and $\mc E$ be graded division superalgebras, with grading groups $G$ and $H$, respectively. If $\D$ is even, then $\D \tensor \mc E$ is a $G\times H$-graded division superalgebra.
\end{lemma}

\begin{proof}
    It is straightforward that the tensor product of graded division algebras is a graded division algebra, hence the result follows from considering $\D$ as a $G$-graded algebra and $\mc E$ as a $H\times \ZZ_2$-graded algebra.
\end{proof}

WE REMIND THE AUTHOR TO EXPLAIN THE DIFFERENT TYPES OF TENSOR PRODUCT IN THE GENERALITIES.

Using the two Lemmas above and Example \ref{ex:FZZ_4-with-superinvoltion}, we have an infinitely many examples of odd graded division superalgebras. Nevertheless, their centers include a $4$-dimensional superalgebra, so none of them is a Type II odd graded division algebra. 
\begin{ex}
    Consider $M_2(\FF)$ with the division grading
\end{ex}

Neither Example \ref{ex:supertransp-graded} or Example \ref{ex:FZZ_4-with-superinvoltion} appear as graded division superalgebras of a Type II grading on $M(n,n)$. But both can be used to construct such superalgebras. We will now show two constructions, one based on Example \ref{ex:supertransp-graded} and other on Example \ref{ex:FZZ_4-with-superinvoltion}.

% Example \ref{ex:FZZ_4-with-superinvoltion} suggests that to have a graded division superalgebra we need to ``increase'' the canonical $\ZZ_2$-grading to a $\ZZ_4$-grading. We have seen two constructions that do that on Section \ref{sec:loop-and-induced-algebras}.% We will start with

Neither Example \ref{ex:supertransp-graded} or Example \ref{ex:FZZ_4-with-superinvoltion} are examples of odd Type II gradings. Nevertheless, we will now present two constructions of such gradings, one based on Example \ref{ex:supertransp-graded} and other based on Example \ref{ex:FZZ_4-with-superinvoltion}. 

\begin{prop}\label{prop:involution-on-cartesian-product}
    Let $R$ be a $G$-graded superalgebra and $\psi: R \to R$ be a super-anti-automorphism (homogeneous of degree $e$). Then $\vphi: R\times R \to R\times R$ geiven by $\vphi(x, y) = (\vphi\inv (y), \vphi (x))$ is a superinvoltion on the $G$-graded superalgebra $R\times R$. \qed
\end{prop}

Using Proposition \ref{prop:involution-on-cartesian-product} with $R$ being the graded superalgebra of Example \ref{ex:supertransp-graded}, we get a graded superalgebra with involution which is not a graded division superalgebra. To make its grading a division grading, we need to refine it. We could do it explicitly, but to not repeat ourselves we will follow another approach.

\begin{ex}
    Let $(R, \psi)$ be the graded superalgebra with super-anti-automorphism of Example \ref{ex:supertransp-graded}. Interpreting its canonical $\ZZ_2$-grading as a $\frac{\ZZ_4}{\langle \bar 2 \rangle}$-grading, its induced algebra $\D = \operatorname{Ind} x$ is a graded division superalgebra. Indeed, $\D = \chi_1 \tensor R \oplus \chi_i \tensor R$ where $\chi_z: \ZZ_4 \to \FF^\times$ denotes the character that sends $\bar 1$ to $z$. Its grading is $\ldots$ .
    
    Now let $\vphi'$ be the homogeneous of degree $e$ map on induced from $\psi$ and $\theta$ be the automorphism given by the action of $\chi_i$. We define $\vphi := \theta \vphi'$. One can readly check that it a superinvolution on $\D$.
\end{ex}

We will now construct a different model of the same superalgebra. This new construction, though, can be done over any field, even if it does not have a square root of $-1$.

%As we saw in Section \ref{ainda-nao-existe}, in an algebraically closed field of characteristic $0$, the induced algebra is always isomorphic to a corresponding loop algebra.

\begin{prop}
    Let $(S, \psi)$ be the superalgebra with superinvolution of Example \ref{ex:FZZ_4-with-superinvoltion} and let $M_2(\FF)$ be the matrix algebra endowed with the $\ZZ_2 \times \ZZ_2$-division grading given by:...
    
    and let $\theta$ be the transposition on it.
\end{prop}



\bibliographystyle{amsalpha}
\bibliography{bibliography}

\end{document}