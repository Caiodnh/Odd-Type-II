\documentclass{amsbook}
\usepackage{basic-preamble}
\overfullrule=2cm

% begin checklist
\usepackage{pifont}
\newcommand{\cmark}{\ding{51}}%
\newcommand{\xmark}{\ding{55}}%

\newcommand{\done}{\rlap{\raisebox{1pt}{\large\hspace{-5pt}\cmark}}}

\newcommand{\cancel}{\rlap{\raisebox{0pt}{\large\hspace{-5pt}\xmark}}}
%\newcommand{\donebox}{\rlap{$\square$}{\raisebox{2pt}{\large\hspace{1pt}\cmark}}%
% end checklist



\newcommand\numberthis{\addtocounter{equation}{1}\tag{\theequation}}

\begin{document}

% ==================================================
% Header
% ==================================================

% Authors
% -------

\author[Caio]{Caio}
\address{Department of Mathematics and Statistics,
	Memorial University of Newfoundland,
St. John's, NL, A1C5S7, Canada}
\email{cdnh22@mun.ca}

% Footnotes
% ---------


% Title
% -----

\date{}

\title{Odd Type II}

% Classification and key words
% ----------------------------

\subjclass[2010]{Primary 17B70; Secondary 16W50, 16W55, 17A70}
\keywords{Graded algebra, associative superalgebra, simple Lie superalgebra, classical Lie superalgebra}

% Abstract
% --------

% Maketitle
% ---------

%\maketitle

\chapter{Temporary Chapter for References}
    
\section{Old chapter on $\vphi$-gradings}

The orthosymplectic and the parapletic superalgebras are defined using a nondegenerate bilinenar form. More specifically, if $\vphi$ is the superadjunction related to this form, then we consider the Lie subsuperalgebra of the skew elements with respect to $\vphi$ (see Subsection {\tt ??}). For the remainder of the chapter, $G$ is a fixed abelian group.

We are now going to define the right notion of ``Type I gradings'' for such algebras %(which, as we are going to see, turn out to be all gradings on then).
The problem here is that not all gradings on the associative superalgebra restrict to the Lie superalgebra of skew elements.
%But the restriction actually work if we have a graded superalgebra with superinvolution, \ie, if the the super involtion is homegeneous of degree $e$.

\begin{defi}
    Let $R$ be a superalgebra and let $\vphi: R\to R$ be any map. We call the pair $(R, \vphi)$ a \emph{superalgebra with distinguished map}. If $\vphi$ is a super-anti-automorphism or an superinvolution, we will call it a \emph{superalgebra with super-anti-automorphism} or \emph{superalgebra with superinvolution}, respectively.
\end{defi}

\begin{defi}
    Let $(R, \vphi)$ and $(S, \psi)$ be superalgebras with distinguished map. A \emph{homomorphism} them is a superalgebra homomorphism $f: R\to S$ such that $f\circ \vphi = \psi \circ f$. In particular, the \emph{automorphism group} of $(R, \vphi)$, denoted by $\Aut(R, \vphi)$, consists of all automorphisms of $R$ commuting with $\vphi$.
\end{defi}

We will be more interested in the case $\vphi$ is super-anti-automorphism (or, more specifically, a superinvolution), but the general case is useful to avoid redundancy in the results and to connect regular gradings to $\vphi$-gradings (see Propositions \texttt{?? [bijections between gradings]  and ?? [example of it]}). Note that if the distinguished maps are taken to be the identity, the concepts above reduce to superalgebra and superalgebra homomorphism. 

\begin{defi}
    Let $(R, \vphi)$ be an associative superalgebra with distinguished map. A \emph{$\vphi$-grading} on it is a $G$-grading $\Gamma$ ($G$ is fixed) such that $\vphi$ is homogeneous of degree $e$. In this case, we say that $(R, \vphi)$ equipped with $\Gamma$ is a \emph{graded superalgebra with distinguished map}.
\end{defi}

Again, the concept of $\vphi$-grading reduces to regular gradings when $\vphi = \id$. In the case $\vphi$ is a super-anti-automorphism, the sub-Lie-superalgebra $\Skew (R, \vphi) = \{ r\in R \mid \vphi(r) = -r \}$ is a graded subsuperspace of $R$. This follows from the lemma bellow:

\begin{lemma}
    Let $V$ be a $G$-graded vector space and let $f: V \to V$ be a homogeneous map of degree $e$. Then the eigenspaces of $f$ are graded subspaces.
\end{lemma}

\begin{proof}
    Let $v\in V$ be an eigenvector with eigenvalue $\lambda \in \FF$ and write $v = \sum_{g\in G} v_g$, where $v_g \in V_g$. On one hand, $f(v) = \lambda v = \sum_{g\in G} \lambda v$. On the other hand, $f(v) = \sum_{g\in G} f(v_g)$. Since the decomposition on homogeneous elements is unique, we have that $f(v_g) = \lambda v_g$ for all $g\in G$, completing the proof.
\end{proof}

\section{$\vphi$-gradings and $\widehat G$-actions}

It is important to know how to translate this notion to the language of $\widehat G$-actions.

\begin{prop}
    Let $(R, \vphi)$ be a superalgebra with distinguished map and consider the usual correspondence between gradings on $R$ and $\dual G$-actions. Then gradings on $(R, \vphi)$ correspond to $\dual G$-actions by automorphisms of $(R, \vphi)$.
\end{prop}

\begin{proof}
    Let $\Gamma$ be a grading on $R$. The map $\vphi$ is homogeneous of degree $e$ if, and only if, $\vphi$ being $\widehat G$-equivariant (Proposition \ref{prop:homogeneous-deg-e-widehatG}), which means $\chi\cdot \vphi (r) = \vphi (\chi\cdot r)$ for all $\chi \in \widehat G$ and all $r \in R$. But this is equivalent to say that $\eta_\Gamma(\chi)$ commutes with $\vphi$ for every $\chi \in \dual G$ or, in other words, that $\eta_\Gamma(\chi) \in \Aut (R, \vphi)$. 
\end{proof}

\begin{prop}\label{prop:homogeneous-deg-e-widehatG}
    Let $V$ and $W$ be graded vector spaces. A linear map $f: V \to W$ is homogeneous of degree $e$ if, and only if, $f$ is a $\widehat G$-equivariant map.
\end{prop}

\section{Relation to gradings on superalgebras of type $A$}

% As we have seem (Section {\tt ??}), all gradings on $M(m+1, n+1)$ restrict to gradings on $A(m, n)$, given us the Type I gradings. Our strategy to obtain the Type II gradings is to embed $\gl(m+1 | n+1)$ in a different associative superalgebra, such that these gradings are restrictions of gradings on the associative superalgebra.

Our goal here is to construct a different model of $\gl(m+1 | n+1)$. More precisely, we want to embed $\gl(m+1 | n+1)$ in an associative superalgebra $\mc R$ such that every Type II grading on $A(m,n)$ is induced from some grading on $\mc R$.

Let $R = M(m+1, n+1)$ and let $\psi: R\to R$ be any super-anti-automorphism. Then consider the superalgebra $\mc R = R\times R$. The map $\vphi: \mc R \to \mc R$ given by $\vphi(x,y) = (\psi\inv(y), \psi(x))$ is a superinvolution on $\mc R$. Then the Lie superalgebra $\Skew (\mc R, \vphi) = \{(r, -\psi(r) | r\in R\}$ is isomorphic to $R^{(-)} = \gl(m+1, n+1)$ by the projection on the first component $\mc R = R\times R \to R$.

\begin{remark}
    For any superalgebra $R$, we could consider the superalgebra $\mc R = R \times R\sop$ and the superinvolution $\vphi: \mc R \to \mc R$ defined by $\vphi(x, y) = (y, x)$. The construction above reduces to this one if we identify $R\sop$ with $R$ via the super-anti-automorphism $\psi$. 
\end{remark}

We can choose any super-anti-isomorphism for $\psi$. One choice would be the supertransposition, but we are going to use a different one that will simplify some computations later. Let $\psi: R \to R$ be the map given by:
\[
    \psi\left(
    \begin{pmatrix}
        a & b\\
        c & d
    \end{pmatrix}\right) = 
    \begin{pmatrix}
        a\transp & i c\transp\\
        i b\transp & d\transp
    \end{pmatrix},
\]
where $i$ is a square root of $-1$.

\begin{remark}
    If we take the basis $\{f_1, \ldots, f_k\}$ on $U^*$ given by $f_j = e_j $ if $|j| = \overline 0$ and $f_j = ie_j$ if $|j| = \overline 1$, then the matrix $\psi (M)$ represents the superadjoint of the operator represented by $M$. Also, $(R, \psi)$ is isomorphic to $R$ with the supertransposition.
\end{remark}

\begin{prop}
    Gradings on $A(m,n)$ are in bijection with $\vphi$-gradings on $\mc R$.
\end{prop}

\begin{proof}
    {\tt needed}
\end{proof}

As we see, not only the Type II gradings on $A(m,n)$ can be recovered this way, but also the Type I gradings. But we will continue to use our previous model to describe the Type I gradings. This is because Type I gradings do not come from $\vphi$-gradings on $\mc R$ making it simple as a graded algebra (Proposition \ref{prop:typeII-iff-RxR-simple}, bellow), so we cannot use Theorem {\tt ??}. The situation is the opposite for Type II gradings.

\begin{prop}\label{prop:typeII-iff-RxR-simple}
    A $\vphi$-grading on $\mc R$ makes it simple as a graded algebra if, and only if, it restricts to a Type II grading.
\end{prop}

\begin{proof}
    Note that $\mc R$ has only two nontrivial ideals, $I = R\times 0$ and $J = 0\times R$.
    
    Let $\Gamma$ be a grading on $\mc R$ and suppose $I$ is a graded ideal. Then $I$ is a graded superalgebra and, hence, $\epsilon_1 := (1,0)$ is homogeneous of degree $e$. Since $(1,1)$ is also homogeneous of degree $e$, so it is $\epsilon_2:= (0,1)$. Then $J = \epsilon_2 \mc R$ is also a graded ideal. By symmetry, we get that $(\mc R, \Gamma)$ is not graded simple if, and only if, $\epsilon_1$ is homogeneous (of degree $e$).
    
    Note that if $\epsilon_1$ is homogeneous of degree $e$, then the projection $\mc R = R\times R \to I$ is homogeneous of degree $e$ and its restriction to $\Skew(\mc R, \vphi)$ is an isomorphism, which implies that the grading on $\gl(m +1, n+1)$ can be obtained by a restriction of a grading on the associative superalgebra $I = R\times 0$.
    
    On the other hand, given a Type I grading on $A(m,n)$, it comes from a grading on $R$. If we consider the grading on $\mc R = R\times R$ by taking the direct sum of graded algebras {\tt (Definition ??)}, then $\mc R$ is not a simple graded algebra.
\end{proof}

\section{Temporary section for references}

Duals of graded supermodules are introduced for the proof of isomorphism theorem of Type I gradings on $A$.

\begin{convention}\label{conv:maps-left-right}
    Whenever it is necessary, we will distinguish maps written on the left (called simply as \emph{maps on the left}) from maps written on the  right (\emph{maps on the right}) as different concepts.
    
    If $\U$ is a left (right) $\mc R$-supermodule, we use the convention of writing the $\mc R$-linear maps on the right (left).
\end{convention}



The following is the converse of the Graded Density Theorem.

\begin{prop}\label{prop:converse-density-thm}
    Let $\D$ be a graded division superalgebra and let $\U$ be a finite dimensional graded right $\D$-supermodule. Consider $\mc R := \End_\D (\U)$. Then $\U$ is simple as a left $\mc R$-supermodule and $\D \iso \End_{\mc R} (\U)$ via the $\D$-action.
    (we will use this isomorphism to identify $\D$ with $\End_{\mc R} (\U)$). 
\end{prop}

\begin{proof}
Let $u \in \U$ be a nonzero vector and complete it to a basis $\{u\} \cup \{ u_\lambda\}_{\lambda \in \Lambda}$, where $\Lambda$ is an indexing set.

Let $v\in \U$ be any element. We can then define $r\in \End_\D (\U)$ by $r(u) = v$ and $f(u_\lambda) = 0$ for all $\lambda \in \Lambda$. Since $u \neq 0$ was arbitrary, this shows $\U$ is a simple $\mc R$-module. 

The statement that the $\mc R$ action is $\D$-linear is, clearly, the same of the $\D$-action being $\mc R$-linear: $r(vd) = (rv)d$ for all $r\in \mc R$ and $d \in \D$. So the $\D$-action give us, indeed, a map $\D \to \End_{\mc R} (\U)$.
Since $\D$ is graded simple and its action is nonzero ($v\cdot 1 = v$ for all $v \in V$), this map is injective.

To show it is surjective, consider $r$ as defined above, but with $v = u$. Let $f\in \End_{\mc R} (U)$. Since $\U$ is a simple $\mc R$-supermodule, $f$ is determined by its value on $u$.
We have that $uf = u d + \sum_{\lambda \in \Lambda} u_\lambda d_\lambda$ for $d, d_\lambda \in \D$ (where all but finitely many $d_\lambda = 0$). 
Them $ uf = (ru)f = r(uf) = r(ud + \sum_{\lambda \in \Lambda} u_\lambda d_\lambda )  = ud$. Hence $f = d$, concluding the proof.
\end{proof}

% ---------------------------------------------------
\section{Superdual of a graded module}\label{ssec:superdual}
% ---------------------------------------------------

We will need the following concepts. Let $\D$ be an associative superalgebra with a grading by an abelian group $G$, so we may consider $\D$ graded by the group $G^\# = G\times \ZZ_2$. Let $\U$ be a $G^\#$-graded \emph{right} $\D$-module. The parity $|x|$ of a homogeneous element $x \in \D$ or $x\in \U$ is determined by $\deg x \in G^\#$. The \emph{superdual module} of $\U$ is $\U\Star = \Hom_\D (\U,\D)$, with its natural $G^\#$-grading and the $\D$-action defined on the \emph{left}: if $d \in \D$ and $f \in \U \Star$, then $(df)(u) = d\, f(u)$ for all $u\in \mc U$.

We define the \emph{opposite superalgebra} of $\D$, denoted by $\D\sop$, to be the same graded superspace $\D$, but with a new product $a*b = (-1)^{|a||b|} ba$ for every pair of $\ZZ_2$-homogeneous elements $a,b \in \D$. The left $\D$-module $\U\Star$ can be considered as a right $\D\sop$-module by means of the action defined by $f\cdot d := (-1)^{|d||f|} df$, for every $\ZZ_2$-homogeneous $d\in \D$ and $f\in \U\Star$.

\begin{lemma}\label{lemma:Dsop}
	If $\D$ is a graded division superalgebra associated to the pair $(T,\beta)$, then $\D\sop$ is associated to the pair $(T,\beta\inv)$.\qed
\end{lemma}

If $\U$ has a homogeneous $\D$-basis $\mc B = \{e_1, \ldots, e_k\}$, we can consider its \emph{superdual basis} $\mc B\Star = \{e_1\Star, \ldots, e_k\Star\}$ in $\U\Star$, where $e_i\Star : \U \rightarrow \D$ is defined by $e_i\Star (e_j) = (-1)^{|e_i||e_j|} \delta_{ij}$.

\begin{remark}\label{rmk:gamma-inv}
	The superdual basis is a homogeneous basis of $\U\Star$, with $\deg e_i\Star = (\deg e_i)\inv$. So, if $\gamma = (g_1, \ldots, g_k)$ is the $k$-tuple of degrees of $\mc B$, then $\gamma\inv = (g_1\inv, \ldots, g_k\inv)$ is the $k$-tuple of degrees of $\mc B\Star$.
\end{remark}

For graded right $\D$-modules $\U$ and $\V$, we consider $\U\Star$ and $\V\Star$ as right $\D\sop$-modules as defined above. If $L:\U \rightarrow \V$ is a $\ZZ_2$-homogeneous $\D$-linear map, then the \emph{superadjoint} of $L$ is the $\D\sop$-linear map $L\Star: \V\Star \rightarrow \U\Star$ defined by $L\Star (f) = (-1)^{|L||f|} f \circ L$. We extend the definition of superadjoint to any map in $\Hom_\D (\U, \V)$ by linearity.

\begin{remark}
	In the case $\D=\FF$, if we denote by $[L]$ the matrix of $L$ with respect to the homogeneous bases $\mc B$ of $\U$ and $\mc C$ of $\V$, then the supertranspose $[L]\sT$ is the matrix corresponding to $L\Star$ with respect to the superdual bases $\mc C\Star$ and $\mc B\Star$.
\end{remark}

We denote by $\vphi: \End_\D (\U) \rightarrow \End_{\D\sop} (\U\Star)$ the map $L \mapsto L\Star$. It is clearly a degree-preserving super-anti-isomorphism. It follows that, if we consider the Lie superalgebras $\End_\D (U)^{(-)}$ and $\End_{\D\sop} (U\Star)^{(-)}$, the map $-\vphi$ is an isomorphism.

We summarize these considerations in the following result:

\begin{lemma}\label{lemma:iso-inv}
	If $\Gamma = \Gamma(T,\beta,\gamma)$ and $\Gamma' = \Gamma(T,\beta\inv,\gamma\inv)$ are $G$-gradings (considered as $G^\#$-gradings) on the associative superalgebra $M(m,n)$, then, as gradings on the Lie superalgebra $M(m,n)^{(-)}$, $\Gamma$ and $\Gamma'$ are isomorphic via an automorphism of $M(m,n)^{(-)}$ that is the negative of a super-anti-automorphism of 	$M(m,n)$.
\end{lemma}

\begin{proof}
	Let $\D$ be a graded division superalgebra associated to $(T,\beta)$ and let $\U$ be the graded right $\D$-module associated to $\gamma$. The grading $\Gamma$ is obtained by an identification $\psi: M(m, n) \xrightarrow{\sim} \End_\D (\U)$. By Lemma \ref{lemma:Dsop} and Remark \ref{rmk:gamma-inv}, $\Gamma'$ is obtained by an identification $\psi': M(m, n) \xrightarrow{\sim} \End_{\D\sop} (\U\Star)$. Hence we have the diagram:

	\begin{center}
		\begin{tikzcd}
			& \End_\D (\U) \arrow[to=3-2, "-\vphi"]\\
			M(m, n) \arrow[ur, "\psi"] \arrow[dr, "\psi'"]\\
			& \End_{\D\sop} (\U\Star)
		\end{tikzcd}
	\end{center}

	Thus, the composition $(\psi')\inv \, (-\vphi) \, \psi$ is an automorphism of the Lie superalgebra $M(m,n)^{(-)}$ sending $\Gamma$ to $\Gamma'$.
\end{proof}

\setcounter{chapter}{3}

\chapter{Gradings on Associative Superalgebras with Superivolution}

The orthosymplectic and the paraplectic Lie superalgebras are defined using a nondegenerate bilinenar form on a superspace. 
More specifically, if $\vphi$ is the superadjunction corresponding to this form, then we consider the subsuperalgebra of the general linear Lie superalgebra consisting of the skew elements with respect to $\vphi$ (see Subsection {\tt ??}). 

\vspace{5mm}
{\tt (More intro to come...)}


{\tt Mention that we are developing the approach of Elduque (J. of \\Al\-ge\-bra 2010, see also Chapter 2 in Monograph) in the super setting.}

\begin{itemize}
    \item Cite Bahturin and Tvaladze; maybe Tvaladze and Tvaladze;
    \item Cite Bahturin and Shestakov;
    \item Also Carlos G\'omes Ambrose ``On the Lie Structure of the Skew Elements of a simple Superalgebra with Superinvolution''
    \item And Racine's paper on superinvolution simple superalgebras.
    \item How what we are doing is different?
    \item Analogous to Lie over $\RR$ (Adri\'an); not literally using results, but using ideas.
\end{itemize}
% For the remainder of the chapter, $G$ is a fixed abelian group.

% In this chapter we are going to describe 

\section{Super-anti-automorphisms and sesquilinear forms}\label{sec:super-anti-auto-and-sesquilinear}

Let $\D$ be a graded division superalgebra and let $\U$ be a nonzero graded right $\D$-supermodule of finite rank. 
Consider the graded superalgebra $R := \End_\D(\mc U)$. 
Then $\mc U$ is an $(R, \D)$-superbimodule. 
By Proposition \ref{prop:converse-density-thm}, $\mc U$ is a simple graded left supermodule over $R$ and we have a natural identification between $\D$ and $\End_R(\mc U)$.

Recall that for a superalgebra $S$, the superalgebra $S\sop$ is defined by $S\sop = S$ as superspaces, but with a different product. For every element $s\in S$, we denote it by $\bar s$ when regarded as an element of $S\sop$, and the product on $S\sop$ is defined by $\bar a \overline {B} = \sign{a}{b} \overline{ba}$ for all $a, b \in S$.
% \begin{notation}
%     Let  be . As sets, we have that $S = S\sop$, but given $s\in S$, we will it by $ \bar s$ when regarded as an element of $S\sop$.
% \end{notation}

As we saw on {\tt Section ??}, $\mc U\Star = \Hom_\D (\mc U, \D)$ is a graded left $\D$-supermodule through $(df)(u) = df(u)$, for all $d\in \D$, $f\in \U\Star$ and $u\in U$. 
We can, then, regard $\U\Star$ as a graded right $\D\sop$-supermodule, through $f \bar d = \sign{f}{d} df$. 
Also, we can make $\mc U\Star$ a graded right $R$-supermodule by defining $(f r) (u) := f(r u)$ for all $f\in \mc U\Star$, $r\in R$ and $u\in \mc U$ (i.e., $f r = f \circ r$, since $R = \End_\D(\mc U)$). 
Hence, we can consider $\mc U\Star$ as a graded left $R\sop$-supermodule via $\bar r f = (-1)^{|r||f|} f r$, % (i.e., $r\cdot f = r\Star (f)$)
and $\mc U\Star$ becomes a $(R\sop, \D\sop)$-superbimodule.

\begin{lemma}\label{lemma:U-star-R-sop}
    The left $R\sop$-supermodule $\mc U\Star$ is graded simple, and the action of $\D\sop$ gives an isomorphism $\D\sop \to \End_{R\sop}(\mc U\Star)$.
\end{lemma}

\begin{proof}
    By definition of superadjoint operator, the above $R\sop$-action corresponds to the representation $\rho\from R\sop \to \End_{\D\sop} (\mc U\Star)$ given by $ \rho(r) = r\Star$, which is an isomorphism since $\U$ is of finite rank. 
    If we use $\rho$ to identify $R\sop$ with $\End_{\D\sop} (\mc U\Star)$, the result follows from Proposition \ref{prop:converse-density-thm}.
\end{proof}

%Note that the $R\sop$-action correspond to the representation $\rho: R\sop \to \End_{\D\sop} (\mc U\Star)$ given by $ \rho(r) = r\Star$, which is an isomorphism. If we use it to identify $R\sop$ with $\End_{\D\sop} (\mc U\Star)$, we can then apply Proposition \ref{prop:converse-density-thm} again and conclude that $\mc U\Star$ is a graded simple left $R\sop$-module and that we naturally can identify $\D\sop$ with $\End_{R\sop}(\mc U\Star)$.

Now let $\vphi: R\to R$ be a super-anti-automorphism that preserves the $G$-degree (\ie, is homogeneous of degree $e$ with respect to the $G$-grading).
We can see it as an isomorphism $R\to R\sop$, $r \mapsto \overline{ \vphi(r)}$, and use it to identify $R$ with $R\sop$. 
In particular, we will make the left $R\sop$-action on $\U\Star$ into a left $R$-action via $r\cdot f = \overline{\vphi(r)}f$, for all $r\in R$ and all $f\in \U\Star$. 
In other words,
%
\begin{equation}\label{eq:R-action-back-on-the-right}
    r\cdot f = \sign{r}{f} f \circ \vphi(r) .
\end{equation}

We will now consider $\U\Star$ as an $(R,\D\sop)$-superbimodule. 
In particular, the superalgebra $\End_R(\U\Star)$ should be understood as the set of $R$-linear maps with respect to the $R$-action on the left (which is the same set as $\End_{R\sop} (\U\Star)$). 
Also, since the action is on the left, we will follow the convention of writing $R$-linear maps on the right.

From Lemma \ref{lemma:U-star-R-sop}, it follows that $\U\Star$ is a simple graded left $R$-supermodule and that we can identify $\D\sop$ with $\End_R (\U\Star) = \End_{R\sop} (\U\Star)$. 
But from Proposition {\tt ??}, $R$ has only one graded simple supermodule up to isomorphism and shift, hence there is an invertible $R$-linear map $\vphi_1: \mc U \to \mc U\Star$ which is homogeneous of some degree $(g_0, \alpha)\in G^\#$. 
Fix one such $\vphi_1$.
% This map is not unique, but we are going to fix one for now. 

\begin{lemma}\label{lemma:nonuniqueness-of-vphi1}
    A map $\vphi_1' : \U \to \U\Star$ is $R$-linear and homogeneous with respect the $G^\#$-grading if, and only if, there is an element $\bar d\in \D\sop = \End_R (\U\Star)$ homogeneous with the respect the $G^\#$-grading such that $ \vphi_1' = \vphi_1 \bar d$, where juxtaposition represents composition of maps written on the right.
\end{lemma}

\begin{proof}
    This follows from the fact that $ \vphi_1\inv \vphi' \in  \D\sop = \End_R(\U\Star)$ if, and only if, $\vphi' \in \Hom_R(\U, \U\Star)$.
\end{proof}

% \begin{proof}
%     Let $\vphi_1'$ be an homogeneous $R$-linear map. Clearly, $\vphi_1\inv \vphi_1' \in \End_R(\U\Star)$ $ = \D\sop$ (note that we are composing functions written on the right). Then $\vphi_1' = \vphi_1 d$, for some homogeneous $d\in \D\sop$, and hence, by the definitions of the right $\D\sop$-action and the left $\D$-action, $\vphi' (u) = \sign{\vphi_1}{d} d\vphi(u)$ for all $u\in \U$. The converse is a direct computation.
% \end{proof}

We will use 
the $R$-linear map 
$\vphi_1$ to construct a super-anti-automorphism $\vphi_0$ on $\D$ and a nondegenerate $\vphi_0$-sesquilinear form $B$ on $\U$. 

\begin{defi}\label{def:sesquilinear-form}
    Let $\vphi_0: \D \to \D$ be a degree preserving su\-per\--an\-ti\--auto\-mor\-phism. We say that a map $B\colon \U \times \U \to \D$ is a \emph{$\vphi_0$-sesquilinear form on $\U$} if it is $\FF$-bilinear, $G^\#$-homogeneous if considered as a linear map $\U\tensor \U \to \D$ and, for all $u,v \in \U$ and $d\in \D$, satisfies
    %
    \begin{enumerate}[(i)]
        \item $B(u,vd) = B(u,v)d$; \label{enum:linear-on-the-second}
        \item $B(ud, v) = (-1)^ {(|B| + |u|)|d|}\vphi_0(d) B(u, v)$. \label{enum:vphi0-linear-on-the-first}
    \end{enumerate}
    
    The \emph{(left) radical} of $B$ is the set $\rad B \coloneqq \{u\in \U \mid B(u, v) = 0 \text{ for all } v\in \U\}$. We say that the form $B$ is \emph{nondegenerate} if $\rad B = 0$.
\end{defi}

Suppose for now that a degree preserving su\-per\--an\-ti\--auto\-mor\-phism $\vphi_0: \D \to \D$ is given. 
We can use it to define a right $\D$-action on $\U\Star$ by interpreting it as an isomorphism from $\D$ to $\D\sop$ and putting 
%
\begin{equation}\label{eq:right-D-action}
    f\cdot d \coloneqq f \overline{\vphi_0(d)},
\end{equation}
%
for all $f \in \U\Star$ and $d\in \D$. 
Using this action, we have $\End_\D (\U\Star) = \End_{\D\sop} (\U\Star)$. We also have the following:

\begin{prop}\label{prop:sesquilinear-form-iff-D-linear-map}
    Fix a degree preserving su\-per\--an\-ti\--auto\-mor\-phism $\vphi_0\from \D \to \D$. 
    The $\vphi_0$-sesquilinear forms $B:\U\times\U \to \D$ are in a one-to-one correspondence with the homogeneous $\D$-linear maps $\theta\from \U \to \U\Star$ via $B \mapsto \theta$ where $\theta(u) \coloneqq B(u, \cdot)$, for all $u \in \U$ or, inversely, $\theta \mapsto B$ where $B(u,v) \coloneqq \theta(u)(v)$, for all $u, v\in \U$. 
    Moreover, $B$ is nondegenerate if, and only if, $\theta$ is an isomorphism.
\end{prop}

\begin{proof}
    Suppose $B$ is given. Condition \eqref{enum:linear-on-the-second} of Definition \ref{def:sesquilinear-form} tells us that $\theta$ defined this way is, indeed, a map from $\U$ to $\U\Star$. 
    Also, it is easy to check that $\theta$ is homogeneous of the same parity and degree as $B$.
    
    Recalling the left $\D$-action on $\U\Star$, condition \eqref{enum:vphi0-linear-on-the-first} tells us that, for all $u\in \U$ and $d\in \D$,
    \[
        \theta (ud) = (-1)^{|d|(|\theta|+|u|)}\vphi_0 (d) \theta (u),
    \]
    which, by the definition of the right $\D\sop$-action on $\U\Star$, is equivalent to $\theta(ud) = \theta(u) \overline{\vphi_0(d)}$, \ie, $\theta$ is, indeed, $\D$-linear considering the left $\D$-action on $\U\Star$ given by Equation \eqref{eq:right-D-action}.
    
    To show that the correspondence is bijective, note that all the considerations above can be reversed when, given $\theta\from \U \to \U\Star$, we define  $B(u, v) \coloneqq \theta(u)(v)$.
    
    The ``moreover'' part follows from the fact that $\rad B = \ker \theta$, so the nondegeneracy of $B$ is equivalent to $\theta$ being injective. But $\U$ and $\U\Star$ have the same (finite) rank over $\D$, so by Proposition {\tt ??}, $\theta$ is injective if, and only if, it is bijective.
\end{proof}

% \begin{lemma}\label{lemma:sesquilinear-form-iff-D-linear-map}
%     Consider $\U\Star$ as a left $\D$-supermodule by using $\vphi_0$ as above. Then there is a bijection between the $\D$-linear maps $\theta\from \U \to \U\Star$ and the $\vphi_0$-sesquilinear forms $B:\U\times\U \to \D$ via $\theta \mapsto B$ where $B(u,v) \coloneqq \theta(u)(v)$. Moreover, $\theta$ is an isomorphism if, and only if, $B$ is nondegenerate.
% \end{lemma}

% Condition \eqref{enum:linear-on-the-second} on the Definition above allows us to use $B$ to define a map $\U \to \U\Star$ by $u \mapsto B(u, \cdot)$. If this is invertible, we say that the form $B$ is \emph{nondegenerate}.

% Note that a degree preserving super-anti-automorphism $\vphi_0: \D \to \D$ can be seen as an isomorphism between $\D$ and $\D\sop$ and, hence, one can use it to define a right $\D$-action on $\U\Star$ by $v$

Coming back to our map $\vphi_1\from \U \to \U\Star$ and using the identifications $\D = \End_R(\U)$ and $\D\sop = \End_R(\U\Star)$ introduced above, consider the map $\D \to \D\sop$ sending $d \mapsto \sign{d}{\vphi_1}\vphi_1\inv d \,\vphi_1$, where juxtaposition denotes composition of maps on the right. 
It is straightforward to check that this map is an isomorphism and, hence, we can consider it as a super-anti-automorphism $\vphi_0 \from \D \to \D$. 
Then, for all $u\in \U$ and $d\in \D$, we have
%
\begin{equation}\label{eq:sesquilinear-before-B}
    (ud)\vphi_1 = u (d\vphi_1) =  u(\vphi_1\vphi_1\inv d \vphi_1) = \sign{d}{\vphi_1}(u\,\vphi_1 )\vphi_0(d).
\end{equation}

% The map $\vphi_0$ depends on the choice o f $\vphi_1$. Following Lemma \ref{lemma:all-possible-vphi1}, if we had started with $\vphi_1' \coloneqq d \vphi_1$ instead, we would have gotten the map $\vphi_0' \coloneqq \vphi_0 \circ \operatorname{sInt}_d$, where $\operatorname{sInt}_d: \D \to \D$ is defined by $\operatorname{sInt}_d(c) = \sign{d}{c} d\inv c d$.

\begin{defi}\label{def:change-map-to-the-left}
    Let $V$ and $V'$ be left $R$-supermodules and let $\psi: V \to V'$ be an $R$-linear map. We define $\psi^\circ: V \to  V'$ to be the following map, written on the left:
    \[
        \psi^\circ(v) = \sign{\psi}{v} v\psi.
    \]
\end{defi}

For example, using the identification $\D\sop = \End_R (\U\Star)$ as before, the left $\D$-action on $\U\Star$ is given by $df = \bar d^\circ (f)$, for all $d\in \D$ and $f\in \U\Star$.

\begin{lemma}\label{lemma:change-of-side-properties}
    Under the conditions of Definition \ref{def:change-map-to-the-left}, we have that, for all $r\in R$ and $v\in V$,  $\psi^\circ (rv) = \sign{\psi}{v} r \psi^\circ (v)$. 
    Further, given another $R$-linear map $\tau: V' \to V''$, we have $(\psi\tau)^\circ = \sign{\psi}{\tau} \tau^\circ\psi^\circ$. \qed
\end{lemma}

% \begin{remark}
%     The map $\psi^\circ$ defined above is not, in general, $R$-linear. Instead, it satisfies $\psi^\circ (rv) = \sign{\psi}{v} r \psi^\circ (v)$, for all $r\in R$ and $v\in V$. 
%     Also, given another $R$-linear map $\theta: \V' \to \V''$, we have that $(\psi\theta)^\circ = \sign{\psi}{\theta} \theta^\circ\psi^\circ$.


    % Let $\psi: \V \to \V'$ and $\theta: \V' \to \V''$ be $R$-linear maps between left $R$-supermodules. Then
    % \begin{enumerate}[(i)]
    %     \item $\psi^\circ (rv) = \sign{\psi}{v} r \psi^\circ (v)$, for all $r\in R$ and $v\in V$;
    %     \item $(\psi\theta)^\circ = \sign{\psi}{\theta} \theta^\circ\psi^\circ$.
    % \end{enumerate}
% \end{remark}

Using the notation just introduced, we can rewrite Equation \eqref{eq:sesquilinear-before-B} as follows:
%
\begin{equation}\label{eq:vphi1-circ-is-D-linear}
    \begin{split}
        \vphi_1^\circ (ud) &= (-1)^{|\vphi_1|(|u|+|d|)} (ud)\vphi_1 \\
        &=\sign{\vphi_1}{u}(u\,\vphi_1) \overline{\vphi_0(d)} = 
        \vphi_1^\circ (u) \overline{\vphi_0(d)},
    \end{split}
\end{equation}
%
which means, considering the right $\D$-action defined via Equation \eqref{eq:right-D-action},
% We can use $\vphi_0\colon \D \to \D\sop$ to define a right $\D$-module structure on $\U\Star$ via $f\cdot d \coloneqq f\vphi_0(d)$, for all $f\in \U\Star$ and $d\in \D$. 
% Considering this action, Equation \eqref{eq:vphi1-circ-is-D-linear} tells us 
that $\vphi_1^\circ$ is $\D$-linear. (Note, however, that Lemma \ref{lemma:change-of-side-properties} shows that $\vphi_1^\circ$ is not $R$-linear, in general.)


% Also, recalling the definition of the right $\D\sop$-action on $\U\Star$, note that Equation \eqref{eq:vphi1-circ-is-D-linear} can be rewritten using the left $\D$-action as
%
% \begin{equation}\label{eq:sesquilinear-with-vphi1-circ}
%     \vphi_1^\circ (ud) = (-1)^{|d|(|\vphi_1|+|u|)}\vphi_0 (d)\vphi_1^\circ (u).
% \end{equation}


% We define $\vphi_1 := \tilde\vphi_1^\circ$ and $\vphi_0: \D \to \D$ by $\vphi_0(d): \sign{\vphi_1}{d} \vphi_1 d^\circ \vphi_1\inv$. 
% Note that the left $\D$-action on $\U\Star$ is connected the right $\D\sop$-action by the formula $f\cdot d = d^\circ \cdot f$, hence $\vphi_0$ is an anti-super-automorphism of $\D$. Also, we have that $\vphi_1(ud) = \sign{u}{d} \vphi_1(d^\circ  $

% \begin{lemma}
%     Let $B: \U\times \U\Star$ be a nondegenerate $\vphi_0$-sesquilinear map. Then, given $r\in R$, there is a unique $s\in R$ such that \[
%         B(ru,v) = \sign{r}{u} B(u,sv).
%     \]
%     Further, the map $r\mapsto s$ is a super-anti-automorphism of $R$.
% \end{lemma}

% \begin{proof}
%     Consider $\theta: \U \to \U\Star$ as in Lemma \ref{lemma:sesquilinear-form-iff-D-linear-map}. 
%     We then have that
%     \begin{align*}
%         \theta(ru)(v) &= \sign{r}{u} \theta(u)(sv)\\
%         \theta (ru)(v) &= \sign{r}{u} (\theta (u)\circ s)(v)\\
%         \theta (ru) &= \sign{r}{u} \theta(u) \circ s\\
%         (\theta \circ r) (u) &= (-1)^{(|\theta| + |u|)|s|} s\Star (
%     \end{align*}
% \end{proof}

Now we define $B: \U\times \U \to \D$ by $B(u, v) = \vphi_1^\circ (u)(v)$. 
By Proposition \ref{prop:sesquilinear-form-iff-D-linear-map}, we have that $B$ is a nondegenerate $\vphi_0$-sesquilinear map. 
Using Lemma \ref{lemma:change-of-side-properties} and Equation \eqref{eq:R-action-back-on-the-right}, we have
%
\begin{equation*}
    \begin{split}
        B(ru,v) &= \vphi_1^\circ (ru)(v) = \sign{r}{\vphi_1} \big(r \cdot \vphi_1^\circ (u) \big) (v)\\ &= \sign{r}{\vphi_1} (-1)^{|r|(|\vphi_1| + |u|)} \big(\vphi_1^\circ (u) \circ \vphi(r) \big)(v) \\ &= \sign{r}{u} \vphi_1^\circ (u) \big( \vphi(r)v \big)= \sign{r}{u} B(u,\vphi(r)v).
    \end{split}
\end{equation*}
%\ie, we have that $\vphi$ is the \emph{superadjunction} with respect to $B$.
We have proved one direction of Theorem \ref{thm:vphi-iff-vphi0-and-B}, bellow. 

% \begin{prop}
%     Let $\vphi_0\from \D \to \D$ be a super-anti-automorphism and let $B\from \U\times \U \to \D$ be a nonzero $\vphi_0$-sesquilinear form. 
%     We have that $B$ is nondegenerate if, and only if, there is a unique super-anti-automorphism $\vphi\from R \to R$ such that
%     %
%     \begin{equation}\label{eq:superadjunction}
%         \forall u, v \in \U, \forall r\in R, \quad B(ru,v) = \sign{r}{u} B(u,\vphi(r)v).
%     \end{equation}
%     %
%     Moreover, if $ $
% \end{prop}

To state the theorem, it is convenient to introduce the following notation. 
Recall that, any invertible element $d\in \D$ gives rise to the inner automorphism $\operatorname{Int}_d\from \D \to \D$ defined by $\operatorname{Int}_d (c) \coloneqq dcd\inv$, for all $c\in \D$. 
If we require $d$ to be homogeneous with respect to the $G^\#$-grading, then $\operatorname{Int}_d$ is an automorphism of $\D$ as a graded superalgebra.

\begin{defi}
    Let $d\in \D$ be a nonzero $G^\#$-homogeneous element. 
    We define the \emph{superinner automorphism} $\operatorname{sInt}_d\from \D \to \D$ by $\operatorname{sInt}_d (c) \coloneqq \sign{c}{d} dcd\inv$, for all $c\in \D$.
\end{defi}

% Considered as a $\FF$-linear map $B: \U\tensor \U \to \D$, it is homogeneous of same parity and degree of $\vphi_1$. It is $\vphi_0$-sesquilinear: since $\vphi_1^\circ (u) \in \U\Star$, we have that $\vphi_1^\circ(u)(vd) = \vphi_1^\circ(u)(v)d$, which give us condition \eqref{enum:linear-on-the-second} of Definition \ref{def:sesquilinear-form};
% condition \eqref{enum:vphi0-linear-on-the-first} is a restatement of \eqref{eq:sesquilinear-with-vphi1-circ}. It is nondegenerate since $\rad B = \ker \vphi_1^\circ$, and $\vphi_1^\circ$ is a bijective $\D$-linear map.



% \begin{lemma}
%     Let $\vphi_0\colon \D\to \D$ be a fixed degree preserving su\-per\--an\-ti\--auto\-mor\-phism. 
%     Given re is a bijective correspondence between the $\vphi_0$-sesquilinear maps $B$ is in bijection with the set of homogeneous maps $\theta\colon \U \to \U\Star$ such that $\theta (ud) = \theta(u) \vphi_0(d)$.
% \end{lemma}

% \begin{proof}
%     Let $B\colon \U \times \U \to \D$ be a $\vphi_0$-sesquilinear form. Define $\theta(u)  \coloneqq B(u, \cdot)$. By condition \eqref{enum:linear-on-the-second} in Definition \ref{def:sesquilinear-form}, the map $\theta (u) \in \U\Star$ for all $u\in \U$. Note that $\theta$ have the same degree and parity of $B$. Hence, the condition \eqref{enum:vphi0-linear-on-the-first} tells us that $\theta(ud) = (-1)^ {(|\theta| + |u|)|d|} \vphi_0(d) \theta(u)$, which, by the Definition of the right $\D\sop$-action on $\U\Star$, becomes $\theta(ud) = \theta (u) \vphi_0 (d)$. 
% \end{proof}

% In particular, we can recover $\vphi$ from the pair $(\vphi_0, B)$.



% One should note that the $\vphi_0: \D \to \D\sop$ give $\U\Star$ a right $\D$-module structure by defining $f\cdot d = f\vphi_0(d)$, for all $f\in \U\Star$ and $d\in \D$. In this sense, from

% Clearly, the $\vphi_0$-sesquilinear form $B$ is \emph{nondegenerate} in the sense that the map $\U \to \U\Star$ given by $u \mapsto B(u, \cdot)$ is an isomorphism of right $\D$-modules. 

\begin{thm}\label{thm:vphi-iff-vphi0-and-B}
    Let $\D$ be a graded division superalgebra and let $\U$ be a nonzero right graded module of finite rank over $\D$. 
    If $\vphi$ is degree preserving super-anti-automorphism on $R \coloneqq \End_R(\U)$, then there is a pair $(\vphi_0, B)$, where $\vphi_0$ is a degree preserving super-anti-automorphism on $\D$ and $B\from \U \times \U \to \D$ is a nondegenerate $\vphi_0$-sesquilinear form, such that
    %
    \begin{equation}\label{eq:superadjunction}
        \forall r\in R\even \cup R\odd,\,\forall u, v \in \U\even \cup \U\odd,  \quad B(ru,v) = \sign{r}{u} B(u,\vphi(r)v).
    \end{equation}
    %
    Conversely, given a pair $(\vphi_0, B)$ as above, there is a unique degree preserving super-anti-automorphism $\vphi$ on $R$ satisfying Equation \eqref{eq:superadjunction}. 
    Moreover, another pair $(\vphi_0', B')$ determines the same super-anti-automorphism $\vphi$ if, and only if, there is a nonzero $G^\#$-homogeneous element $d\in \D$ such that $\vphi_0 = \operatorname{sInt}_d \circ \vphi_0$ and $B'(u, v) = dB (u, v)$ for all $u, v \in \U$.
\end{thm}

\begin{proof}
    The first assertion is already proved. For the converse, let $\vphi_0$ be a degree preserving super-anti-automorphism on $\D$, let $B\from \U \times \U \to \D$ be a nondegenerate $\vphi_0$-sesquilinear form and consider $\theta$ as in 
    Proposition \ref{prop:sesquilinear-form-iff-D-linear-map}. Then Equation \eqref{eq:superadjunction} is equivalent to:
    %
    \begin{alignat*}{2}
        \forall r\in R\even \cup R\odd,\,\forall u, v &\in \U\even \cup \U\odd,& \theta (ru)(v)&= \sign{r}{u} \theta(u)(\vphi(r)v)\\
        &&&= \sign{r}{u} \big(\theta (u)\circ \vphi(r)\big)(v)\\
        %
        \intertext{and, hence, equivalent to}
        %
        \forall r\in R\even \cup R\odd,\,&\forall u \in \U\even \cup  \U\odd,&   \theta (ru) &= \sign{r}{u} \theta(u) \circ \vphi(r).
        %
        \addtocounter{equation}{1}\tag{\theequation}\label{eq:theta-is-almost-R-superlinear}\\
        %
        \intertext{Recalling the definition of superadjoint operator, Equation \eqref{eq:theta-is-almost-R-superlinear} becomes}
        %
        \forall r\in R\even \cup R\odd,\,\forall u &\in \U\even \cup \U\odd, & (\theta \circ r) (u) &= \sign{r}{u} (-1)^{(|\theta| + |u|)|r|} \big(\vphi(r)\big)\Star \big(\theta(u)\big) \\
        &&&=  \sign{r}{\theta} \big(\big(\vphi(r)\big)\Star \circ \theta \big) (u),
        %
        \intertext{which is the same as}
        %
        \forall r&\in R\even \cup R\odd, &\theta \circ r &= \sign{r}{\theta}  \big(\vphi(r)\big)\Star \circ \theta.
        \intertext{In other words, we have}
        \forall r&\in R\even \cup R\odd, & \big(\vphi(r)\big)\Star &= \sign{r}{\theta}\, \theta \circ r \circ \theta\inv.
        %
        \addtocounter{equation}{1}\tag{\theequation}\label{eq:vphi-r-Star-is-a-superconjugation}
    \end{alignat*}
    %
    % We have shown that
    % %
    % \begin{equation}
    %   \big(\vphi(r)\big)\Star = \sign{r}{\theta}\, \theta \circ r \circ \theta\inv.
    % \end{equation}
    %
    Since $\U$ has finite rank over $\D$, the superadjunction map $\End_\D (\U) \to \End_{\D\sop} (\U\Star)$ is invertible and, hence, $\vphi$ is uniquely determined.
    Also, the properties of superadjunction imply that $\vphi$ is, indeed, a super-anti-automorphism of $R$.
    
    For the ``moreover'' part, let $d$ be a nonzero $G^\#$-homogeneous element of $\D$ and consider $\vphi_0' = \operatorname{sInt}_d \circ \, \vphi_0$ and $B' = dB$. 
    We have that $B'$ is $\vphi_0'$-sesquilinear since, for all $c\in \D\even \cup \D\odd$ and $u,v \in \U\even \cup \U\odd$,
    %
    \begin{align*}
        B' (uc, v) &= dB (uc, v) = (-1)^{(|B| + |u| ) |c|} d\vphi_0(c) B(u,v) \addtocounter{equation}{1}\tag{\theequation}\label{eq:dB-is-sesquilinear}\\
        &= (-1)^{(|B| + |u| ) |c|} d\vphi_0(c) d\inv d B(u,v) \\
        &=  (-1)^{(|B| + |u| ) |c|} \sign{c}{d} (\operatorname{sInt}_d \circ\, \vphi_0) (c)\, dB(u,v)\\
        &= (-1)^{(|B'| + |u|) |c|} (\operatorname{sInt}_d \circ\, \vphi_0) (c)\, B'(u,v).
    \end{align*}
    %
    To show $B'$ that is nondegenerate, note that $dB(u,v) = 0$ implies $B(u,v) =0$, hence  $\rad B' \subseteq \rad B$. 
    Finally, it is straightforward that Equation \eqref{eq:superadjunction} is still true if we replace $B$ by $B'$. 
    
    %and let $\vphi_0'$ be a super-anti-automorphism on $\D$. 
    % we will first check that $B'$ is, indeed, $\vphi_0'$-sesquilinear.
    
    To prove the other direction, we consider, again, the left $R$-supermodule structure on $\U\Star$ given by Equation \eqref{eq:R-action-back-on-the-right} and let $\theta\from \U \to \U\Star$ be as above, \ie, $\theta(u) = B(u, \cdot)$. Similarly, let $\theta'\from \U \to \U\Star$ be defined by $\theta' \coloneqq B'(u, \cdot)$.
    
    Combining Equations \eqref{eq:R-action-back-on-the-right} and  \eqref{eq:theta-is-almost-R-superlinear}, we have that
    %
    \begin{equation}\label{eq:theta-is-R-superlinear}
        \theta(ru) = \sign{\theta}{r} r\cdot \theta(u).
    \end{equation}
    %
    Define the map $\tilde \theta\from \U \to \U\Star$, written on the right, by $u\, \tilde\theta = \sign{u}{\theta} \theta(u)$, for all $u\in \U$ (compare with Definition \ref{def:change-map-to-the-left} and note that $\theta = (\tilde \theta)^\circ$). 
    Then Equation \eqref{eq:theta-is-R-superlinear} becomes $(ru)\tilde\theta = r \cdot (u\,\tilde\theta)$, \ie, $\tilde\theta$ is $R$-linear.
    
    % Now, given a pair $(\vphi_0', B')$, we have the corresponding map $\theta'(u) \coloneqq B'(u, \cdot)$, for all $u\in \U$. 
    All these considerations about $\theta$ are also valid for $\theta'$, so we define $\tilde\theta'\from \U \to \U\Star$ by $u\,\tilde\theta' \coloneqq \sign{u}{\theta'} \theta'(u)$ and we get another $R$-linear map from $\U$ to $\U\Star$.
    By Lemma~\ref{lemma:nonuniqueness-of-vphi1}, there is $\bar d\in \D\sop$ such that $\tilde\theta' = \tilde\theta \bar d$.
    Applying Lemma \ref{lemma:change-of-side-properties}, this implies \[\theta' = \sign{\theta}{\bar d} \bar d^\circ \theta.\]
    But $\bar d^\circ \theta (u) = d\theta(u)$, where in the last term we use the left $\D$-action on $\U\Star$. 
    Therefore $B'(u,v) = \theta'(u)(v) = d\theta(u)(v) = \sign{\theta}{d} dB(u, v)$, for all $u,v \in \U$. 
    Replacing $d$ by $\sign{\theta}{d} d$, we get $B' = dB$. 
    
    It remains to check that $\vphi_0' = \operatorname{sInt}_d \circ\, \vphi_0$. 
    Since $B' = dB$, Equation \eqref{eq:dB-is-sesquilinear} is valid, hence $B'$ is $(\operatorname{sInt_d \circ \vphi_0})$-sesquilinenar.
    We then have, for all $c\in \D\even \cup \D\odd$ and $u,v \in \U\even \cup \U\odd$,
    \[
        \vphi_0'(c)\, B'(u,v) = (-1)^{(|B'| + |u|) |c|} B'(uc,v) = (\operatorname{sInt_d \circ\, \vphi_0}) (c)\, B'(u,v).
    \]
    The form $B'$ is nondegenerate, so we can choose $G^\#$-homogeneous $u,v\in \U$ with $B'(u,v)\neq 0$. Then $B'(u,v)$ is invertible, hence $\vphi_0' (c) = (\operatorname{sInt}_d \circ\, \vphi_0) (c)$, concluding the proof.
\end{proof}

\begin{convention}\label{conv:pick-B-even}
    Under the conditions of Theorem \ref{thm:vphi-iff-vphi0-and-B}, if $\D$ is an odd graded division superalgebra (\ie, $\D\odd \neq 0$), then we will choose the form $B$ to be even. 
    This is possible since, by the ``moreover'' part, we can substitute an odd form $B$ by $dB$, for some $d\in \D\odd$.
\end{convention}

\section{Lost and Found}

Recall that the center of an algebra $h$

\begin{prop}
    
\end{prop}

% \begin{remark}\label{rmk:we-only-need-B'-nonzero}
%     Note that in the proof of the ``moreover'' part of Theorem \ref{thm:vphi-iff-vphi0-and-B}, we do not need to assume $B'$ is nondegenerate, we only use that it is nonzero.
% \end{remark}

\section{Matrix representation of a super-anti-automorphism}

In this section, we are going to express the super-anti-automorphism $\vphi$ in terms of matrices with entries in $\D$. 
One could do that by following Equation \eqref{eq:vphi-r-Star-is-a-superconjugation}, but we will take a different path. 

As before, we suppose $\D$ is a graded division superalgebra, $\U$ is a nonzero right graded module of finite rank over $\D$, $R = \End_\D (\U)$ and $\vphi$ is a degree preserving super-anti-automorphism on $R$. 
Also, let $\vphi_0$ be a super-anti-automorphism on $\D$ and $B$ be a nondegenerate $\vphi_0$-sesquilinear form on $\U$ determing $\vphi$ as in Theorem \ref{thm:vphi-iff-vphi0-and-B}.

From now on, let $\mc B = \{e_1, \ldots, e_k\}$ be a fixed homogeneous $\D$-basis of $\U$. 
% We will assume that the even elements precede the odd ones. 
If $\D$ is odd (\ie, $\D\odd \neq 0$), we will assume that all elements of $\mc B$ are even, since we can multiply the odd ones by a nonzero homogeneous odd element in $\D$. 
We will use $\mc B$ to identify $R = \End_\D (\U)$ with $M_k (\D)$. 
Also, we will denote $|e_i|$ simply by $|i|$ for all $i \in \{1, \ldots, k\}$. 

\begin{defi}
    Let $X = (x_{ij})$ be a matrix in $M_k (\D)$. 
    We define $\vphi_0 (X)$ to be the matrix obtained by applying $\vphi_0$ in each entry, \ie, $\vphi_0 (X) \coloneqq (\vphi_0(x_{ij}))$. 
    We also extend the definition \emph{supertranspose} {(\tt see ??)} to matrices over $\D$ by putting $X\stransp \coloneqq \big((-1)^{(|i| + |j|) |i|} x_{ji} \big)$. 
    Note that, with our choice of $\mc B$ in the case of odd $\D$, we have that $X\stransp = X\transp$, the ordinary transpose.
\end{defi}

\begin{prop}\label{prop:matrix-vphi}
    Let $\Phi = (\Phi_{ij}) \in M_k(\D)$ be the matrix representing $B$, \ie, $\Phi_{ij} = B(e_i, e_j)$. For every $r\in  R\even \cup R\odd$, let $X \in M_k(\D)$ be the matrix representing $r$ and let $Y \in M_k(\D)$ be the matrix representing $\vphi(r)$.
    Then, following the Convention \ref{conv:pick-B-even} if $\D$ is odd, we have
    %
    \begin{align}
        Y &= \Phi\inv\, \vphi_0( X\stransp )\, \Phi. \addtocounter{equation}{1}\tag{\theequation}\label{eq:matrix-vphi-D-even}
        % \intertext{and, if $\D$ is odd, we have}
        % Y &= \sign{B}{r}\,\Phi\inv\, \vphi_0( X\stransp )\, \Phi.\addtocounter{equation}{1}\tag{\theequation}\label{eq:matrix-vphi-D-odd}
    \end{align}
\end{prop}

\begin{proof}
    First of all, note that Equation \eqref{eq:superadjunction} is equivalent to the following:
    %
    \begin{alignat*}{2}
        \forall e_i, e_j \in \mc B, && B(re_i, e_j) &= \sign{r}{i} B(e_i, \vphi(r) e_j),
    \intertext{which, by the definitions of $X$, $Y$ and $\Phi$, becomes}
        \forall e_i, e_j \in \mc B, && \quad B\bigg(\sum_{\ell=1}^k e_\ell x_{\ell i}, e_j\bigg) &= \sign{r}{i} B\bigg(e_i, \sum_{\ell=1}^k e_\ell y_{\ell j}\bigg) \addtocounter{equation}{1}\tag{\theequation}\label{eq:superadjunction-matrix}.
    \end{alignat*}
    %
    
    Fix arbitrary $p,q \in \{1, \ldots, k\}$ and suppose that the $(p,q)$-entry of $X$ is a nonzero $G^\#$-homogeneous element of $\D$ and $x_{ij} = 0$ elsewhere, \ie, $X$ represents the map $r \in \End_\D(\U)$ defined by $r e_i = \delta_{iq} e_p x_{pq}$. 
    By the $\FF$-linearity of Equations \eqref{eq:matrix-vphi-D-even} and \eqref{eq:matrix-vphi-D-odd}, it suffices to consider such $X$. 
    Note that $|r| = |e_p| + |x_{pq}| - |e_q| = |p| + |q| + |x_{pq}|$. 
    Then, on the one hand,
    %
    \begin{align*}
        B\bigg(\sum_{\ell=1}^k e_\ell x_{\ell i}, e_j\bigg) = B (e_p x_{pi}, e_j) &= (-1)^{ (|B| + |p|) |x_{pi}|} \vphi_0(x_{pi}) B(e_p, e_j)
        \\&= (-1)^{ (|B| + |p|) |x_{pi}|} \vphi_0(x_{pi}) \Phi_{pj},
    %\end{align*}
    %
    \intertext{which is only nonzero if $i = q$. On the other hand,}
    %
    %\begin{align*}
        \sign{r}{i} B\bigg(e_i, \sum_{\ell=1}^k e_\ell y_{\ell j}\bigg)
        &= \sign{r}{i} \sum_{\ell=1}^k B(e_i, e_\ell) y_{\ell j}\\
        &= (-1)^{ (|p| + |q| + |x_{pq}|) |i| } \sum_{\ell=1}^k \Phi_{i \ell} y_{\ell j}. 
    %\end{align*}
    %
    \intertext{Therefore, Equation \eqref{eq:superadjunction-matrix} is equivalent to, for all $i,j \in \{1, \ldots, k\}$,}
    %
    %\begin{align*}
        (-1)^{ (|B| + |p|) |x_{pi}|} \vphi_0(x_{pi}) \Phi_{pj} %&= \sign{r}{i} \sum_{\ell=1}^k \Phi_{i \ell} y_{\ell j}\\
        &= (-1)^{ (|p| + |q| + |x_{pq}|) |i| } \sum_{\ell=1}^k \Phi_{i \ell} y_{\ell j}. \addtocounter{equation}{1}\tag{\theequation}\label{eq:expression}
    \end{align*}
    %

    If $\D$ is even, then $|x_{pi}| = \bar 0$ and this equation reduces to
    \[              \vphi_0(x_{pi}) \Phi_{pj} = (-1)^{ (|p| + |q|) |i| } \sum_{\ell=1}^k \Phi_{i \ell} y_{\ell j}
    \]
    or, equivalently, 
    \[
        \sum_{\ell=1}^k \Phi_{i \ell} y_{\ell j} = (-1)^{ (|p| + |q|) |i| } \vphi_0(x_{pi}) \Phi_{pj}.
    \]
    The left-hand side is the $(i,j)$-entry of $\Phi\, Y$. 
    The right-hand side is only nonzero if $i = q$, so it can be rewritten as 
    $(-1)^{ (|p| + |i|) |i| } \vphi_0(x_{pi}) \Phi_{pj}$. 
    Recalling our choice of $X$, this is equal to $\sum_{\ell =1}^k (-1)^{ (|\ell| + |i|) |i| } \vphi_0(x_{\ell i}) \Phi_{\ell j}$, since $x_{\ell i}$ is only nonzero if $\ell = p$. 
    Hence the right-hand side is the $(i,j)$-entry of $\vphi_0 (X\stransp) \Phi$, and Equation \eqref{eq:matrix-vphi-D-even} follows. 
    
    If $\D$ is odd, by our choice of basis, Equation \eqref{eq:expression} reduces to
    \[
        (-1)^{|B| |x_{pi}|} \vphi_0(x_{pi}) \Phi_{pj} =  \sum_{\ell=1}^k \Phi_{i \ell} y_{\ell j},
    \]
    which, by the same reasoning as above, implies $(-1)^{|B||r|} \vphi_0(X\stransp) \Phi = \Phi Y$. 
    But we are following Convention \ref{conv:pick-B-even}, hence $|B| = \bar 0$ and,therefore, we have the desired result.
\end{proof}

% \begin{proof}
%     First of all, note that Equation \ref{eq:superadjunction} is equivalent to, for all $e_i, e_j \in \mc B$,
%     %
%     \[
%         B(re_i, e_j) = \sign{r}{i} B(e_i, \vphi(r) e_j).
%     \]
%     By the definitions of $X$, $Y$ and $\Phi$, we have that
%     %
%     \begin{align*}
%         B\bigg(\sum_{\ell=1}^k e_\ell x_{\ell i}, e_j\bigg) &= \sum_{\ell=1}^k (-1)^{( |B| + |\ell|) |x_{\ell i}|} \vphi_0( x_{\ell i} ) B (e_\ell x_{\ell i}, e_j)\\ &= \sum_{\ell=1}^k (-1)^{( |B| + |\ell|) |x_{\ell i}|} \vphi_0( x_{\ell i} ) \Phi_{\ell j}
%         \intertext{and}
%         \sign{r}{i} B\bigg(e_i, \sum_{\ell=1}^k e_\ell y_{\ell j}\bigg) 
%         &= \sign{r}{i} \sum_{\ell=1}^k B(e_i, e_\ell) y_{\ell j}\\
%         &= \sign{r}{i} \sum_{\ell=1}^k \Phi_{i \ell} y_{\ell j}. 
%     \end{align*}
%     %
    
%     Let $p,q \in \{1, \ldots, k\}$, and suppose $X$ is such that $x_{pq}$ is a nonzero $G^\#$-homogeneous element of $\D$ and $x_{ij} = 0$ elsewhere, \ie, $X$ represents the map $r \in \End_\D(\U)$ defined by $r e_i = \delta_{iq} e_p x_{pq}$. 
%     By the $\FF$-linearity of Equations \eqref{eq:matrix-vphi-D-even} and \eqref{eq:matrix-vphi-D-odd}, it suffices to consider such $X$. 
%     Note that $|r| = |e_q| + |x_{pq}| - |e_p| = |p| + |q| + |x_{pq}|$. 
%     Then
%     %
%     \begin{align*}
%         B\bigg(\sum_{\ell=1}^k e_\ell x_{\ell i}, e_j\bigg) = B (e_p x_{pi}, e_j) &= (-1)^{ (|B| + |p|) |x_{pi}|} \vphi_0(x_{pi}) B(e_p, e_j)
%         \\&= (-1)^{ (|B| + |p|) |x_{pi}|} \vphi_0(x_{pi}) \Phi_{pj},
%     \end{align*}
%     %
%     which is only nonzero if $i = q$. 
%     Combining this with Equation \eqref{eq:superadjunction-matrix}, we get
%     %
%     \begin{align*}
%         (-1)^{ (|B| + |p|) |x_{pi}|} \vphi_0(x_{pi}) \Phi_{pj} &= \sign{r}{i} \sum_{\ell=1}^k \Phi_{i \ell} y_{\ell j}\\
%         &= (-1)^{ (|p| + |q| + |x_{pq}|) |i| } \sum_{\ell=1}^k \Phi_{i \ell} y_{\ell j}. \addtocounter{equation}{1}\tag{\theequation}\label{eq:expression}
%     \end{align*}
%     %
%     If $\D$ is even, then $|x_{pi}| = 0$ and this expression reduces to
%     \[              \vphi_0(x_{pi}) \Phi_{pj} = (-1)^{ (|p| + |q|) |i| } \sum_{\ell=1}^k \Phi_{i \ell} y_{\ell j}
%     \]
%     or, reordering it, 
%     \[
%         \sum_{\ell=1}^k \Phi_{i \ell} y_{\ell j} = (-1)^{ (|p| + |q|) |i| } \vphi_0(x_{pi}) \Phi_{pj}.
%     \]
%     The left-hand side is the entry $ij$ of $\Phi\, Y$. 
%     The right-hand side is only nonzero if $i = q$, so it can be rewritten as 
%     $(-1)^{ (|p| + |i|) |i| } \vphi_0(x_{pi}) \Phi_{pj}$. 
%     Also, recalling our choice of $X$, it is equal to $\sum_{\ell =1}^k (-1)^{ (|\ell| + |i|) |i| } \vphi_0(x_{\ell i}) \Phi_{\ell j}$, since $x_{\ell i}$ is only nonzero if $\ell = p$. Hence the right-hand side is the $ij$ entry of $\vphi_0 (X\stransp) \Phi$, and Equation \eqref{eq:matrix-vphi-D-even} follows. 
    
%     If $\D$ is odd, by our choice of basis, Equation \eqref{eq:expression} reduces to
%     \[
%         (-1)^{|B| |x_{pi}|} \vphi_0(x_{pi}) \Phi_{pj} =  \sum_{\ell=1}^k \Phi_{i \ell} y_{\ell j},
%     \]
%     which implies $(-1)^{|B||r|} \phi_0(X\stransp) \Phi = \Phi Y$.
% \end{proof}

% \begin{remark}
%     Note that if $\D$ is odd, replacing $B$ by $dB$ with an odd $d$ if necessary, we can guarantee that $B$ is even. 
%     In this case, Equation \eqref{eq:matrix-vphi-D-odd} reduces to \eqref{eq:matrix-vphi-D-even}.
% \end{remark}

% \begin{cor}[of the proof]
%     Let $\vphi_0$ be a super-anti-automorphism on $\D$ and let $B$ be a nonzero $\vphi_0$-sesquilinear form on $\U$. If $\vphi\from R \to R$ be a parity preserving map satisfying Equation \eqref{eq:superadjunction}. 
%     Then $B$ is nondegenerate if, and only if, $\vphi$ is a super-anti-automorphism.
% \end{cor}

\section{Superinvolutions and sesquilinear forms}

Our goal now is to specialize the results of Section \ref{sec:super-anti-auto-and-sesquilinear} to the case where $\vphi$ is a superinvolution. 
To this end, let us investigate what super-anti-automorphism of $\D$ and what sesquilinear form on $\U$ determine the super-anti-automorphism $\vphi\inv$. 
Again, we suppose $\D$ is a graded division superalgebra, $\U$ is a nonzero right graded module of finite rank over $\D$ and put $R = \End_\D (\U)$.

\begin{defi}\label{def:barB}
    Given a super-anti-automorphism $\vphi_0$ on $\D$ and a $\vphi_0$-sesqui\-li\-near form $B$ on $\U$,  we define $\overline {B}\from \U\times \U \to \D$ by $\overline {B} (u,v) \coloneqq \sign{u}{v} \vphi_0\inv (B(v, u))$ for all $u, v \in \U$.
\end{defi}

% \begin{lemma}
%     Let $B\from \U\times \U\Star \to \D$ be a $\vphi_0$-sesquilinear form. Suppose there is a super-anti-automorphism $\vphi$ on $R = \End_\D(U)$ such that Equation \eqref{eq:superadjunction} holds. If $B$ is nonzero, then it is nondegenerate.
% \end{lemma}

\begin{prop}\label{prop:barB-determines-vphi-inv}
    Under the conditions of Definition \ref{def:barB}, we have that $\overline {B}$ is a $\vphi_0\inv$-sesquilinear form of the same degree and parity as $B$. 
    Further, if $B$ is nondegenerate and $\vphi$ is the super-anti-automorphism on $R$ determined by $(\vphi_0, B)$ as in Theorem \ref{thm:vphi-iff-vphi0-and-B}, then $\overline{B}$ is nondegenerate and $\vphi\inv$ is determined by $(\vphi_0\inv, \overline B)$, \ie, 
    %
    \begin{equation}\label{eq:barB-superadjunction}
        \forall r\in R\even \cup R\odd ,\,\forall u, v \in \U\even \cup \U\odd,  \quad \overline {B}(ru,v) = \sign{r}{u} \overline {B}(u,\vphi\inv (r)v).
    \end{equation}
    %
\end{prop}

\begin{proof}
    Since $B$ is $\FF$-bilinear, so is $\overline {B}$. 
    Also, since $\vphi_0$ preserves degree and parity, $\overline {B}$ is homogeneous of the same degree and parity as $B$. 
    Let us check the conditions of Definition \ref{def:sesquilinear-form} and Equation \eqref{eq:barB-superadjunction}.
    \vspace{2mm}
    \begin{align*}
       \intertext{ Condition \eqref{enum:linear-on-the-second}:}
        \overline {B} (u,vd) &= (-1)^{|u|( |v| + |d|)} \vphi_0\inv (B(vd, u))  \\
        & = (-1)^{|u|( |v| + |d|)} (-1)^{|d| (|B| + |v|)} \vphi_0\inv \big( \vphi_0(d) B(v, u) \big) \\
        & = (-1)^{|u||v| + |u||d| + |d||B| + |d||v|} (-1)^{|d| (|B| + |v| + |u|) }  \vphi_0\inv \big(B(v, u)) d \\ &= \sign{u}{v} \vphi_0\inv \big(B(v, u)) d = \overline {B}(u, v) d .
    \intertext{Condition \eqref{enum:vphi0-linear-on-the-first}:}
        \overline {B}(ud, v) &= (-1)^{(|u| + |d|) |v|} \vphi_0\inv \big( B(v, ud) \big) \\ &= (-1)^{(|u| + |d|) |v|} \vphi_0\inv \big( B(v, u)d \big) \\ &= (-1)^{(|u| + |d|) |v|} (-1)^{|d| (|B| + |v| + |u|)} \vphi_0\inv (d) \vphi_0\inv\big( B(v, u) \big) \\ &= (-1)^{|u||v| + |d| |B| + |d||u|} \vphi_0\inv (d) \vphi_0\inv\big( B(v, u) \big) \\ &= (-1)^{(|B| + |u|) |d|} \vphi_0\inv (d) \overline {B}(u, v).
      \intertext{For Equation \eqref{eq:barB-superadjunction}, note that replacing $r$ for $\vphi\inv(r)$, Equation \eqref{eq:superadjunction} can be rewritten as}
        B(v, ru) &= \sign{r}{v} B(\vphi\inv(r)v,u).
      \intertext{Hence, we have that}
        \overline {B}(ru, v) &= (-1)^{(|r| + |u|) |v|} \vphi_0\inv \big( B(v, ru) \big)\\ &= (-1)^{(|r| + |u|) |v|} (-1)^{|r||v|} \vphi_0\inv \big( B(\vphi\inv (r)v, u) \big) \\ &= (-1)^{|u||v|} \vphi_0\inv \big( B(\vphi\inv (r)v, u) \big) \\ &= (-1)^{|u||v|} (-1)^{(|r| + |v|) |u|} \overline {B}(u, \vphi\inv (r)v) \\ &= \sign{r}{u} \overline {B}(u, \vphi\inv (r)v).
    \end{align*}
    
    Finally, Equation \eqref{eq:barB-superadjunction} together with $B$ being nondegenerate implies that $\overline{B}$ is nondegenerate. 
    To see that, let $u$ be a nonzero homogeneous element in $\rad \overline{B}$. 
    Then for for every $r\in R\even \cup R\odd$ and $v\in \U\even \cup \U\odd$, we have that $\overline{B}(u, \vphi\inv (r) v) = 0$, hence $\overline{B}(ru, v) = 0$. 
    Since $r \in R\even \cup R\odd$ and $v\in \U\even \cup \U\odd$ were arbitrary, this implies $\overline{B} (Ru, \U) = 0$. 
    But $\U$ is simple as a graded $R$-supermodule, so we would have $\overline{B}(\U, \U) = 0$ and then, using that $\vphi_0$ is bijective, $B (\U, \U) = 0$, a contradiction.
\end{proof}

\begin{lemma}\label{lemma:bar-dB}
    Under the conditions of Definition \ref{def:barB}, let $d$ be a nonzero $G^\#$-homogeneous element of $\D$ and consider $\vphi_0' \coloneqq \operatorname{sInt}_d\circ\, \vphi_0$ and $B' \coloneqq d B$. 
    Then $\overline {B'} = (-1)^{|d|} \vphi_0\inv (d) \overline B$.
\end{lemma}

\begin{proof}
    Note that $(\vphi_0')\inv = \vphi_0\inv \circ \operatorname{sInt}_{d}\inv = \vphi_0\inv \circ \operatorname{sInt}_{d\inv}$. 
    Hence, for all $u, v \in \U\even \cup \U\odd$,
    %
    \begin{align*}
        \overline {B'} (u,v) &= \sign{u}{v} (\vphi_0\inv \circ \operatorname{sInt}_{d\inv})  (d B(v, u) ) \\
        &= \sign{u}{v} \vphi_0\inv \big( (-1)^{|d| (|d| + |B| + |u| + |v|)}\,  d\inv d B (v, u) d \big) \\ &= \sign{u}{v} (-1)^{|d|}\, \vphi_0\inv (d) \vphi_0\inv(B(v, u)) =  (-1)^{|d|}\,\vphi_0\inv (d) \overline {B} (u, v).
    \end{align*}
\end{proof}

We are primarily interested in the case $\FF$ is an algebraically closed field and $\D$ is finite dimensional. 
In this case, we have that $\D\even_e = \FF 1$, so we are under the hypothesis of the following theorem:

\begin{thm}\label{thm:vphi-involution-iff-delta-pm-1}
    % Suppose $\FF$ is an  algebraically closed field. 
    Let $\D$ be a graded division superalgebra such that $\D\even_e = \FF 1$, let $\U$ be a nonzero right graded module of finite rank over $\D$ and let $\vphi$ be a degree preserving super-anti-automorphism on $R \coloneqq \End_\D (\U)$. 
    Consider a super-anti-automorphism $\vphi_0$ on $\D$ and a nondegenerate $\vphi_0$-sesqui\-li\-near form $B$ on $\U$ determining $\vphi$ as in Theorem \ref{thm:vphi-iff-vphi0-and-B}.
    Then $\vphi$ is a superinvolution if, and only if, $\vphi_0$ is a superinvolution and $\overline B = \pm B$. 
\end{thm}

\begin{proof}
    Using Proposition \ref{prop:barB-determines-vphi-inv} and Theorem \ref{thm:vphi-iff-vphi0-and-B}, we conclude that $\vphi = \vphi\inv$ if, and only if, there is $\delta \in \D$ such that (i) $\vphi_0\inv = \operatorname{sInt}_\delta \circ \,\vphi_0$ and (ii) $\overline {B} = \delta B$. 
    % If $\vphi_0^2 =\id_\D$ and $B = \pm \overline{B}$, we have conditions (i) and (ii) sati
    % To prove the ``if'' part of the statement, we take $\delta \in \{\pm 1\}$, since then (i) becomes $\vphi_0 = \vphi_0\inv$ and (ii), $B = \pm \overline{B}$.
    
    If $\vphi_0^2 =\id_\D$ and $\overline{B} = \delta B$ with $\delta \in \{\pm 1\}$, conditions (i) and (ii) are satisfied, proving the ``if'' part of the statement. 

    For the ``only if'' part, since both parity and degree of $B$ and $\overline {B}$ coincide, (ii) implies $\delta \in \D\even_e$. 
    We are assuming $\D\even_e = \FF 1$, so $\operatorname{sInt}_\delta = \id_\D$ and (i) becomes $\vphi_0 = \vphi_0\inv$, \ie, $\vphi_0$ is a superinvolution. 
    Also from $\delta \in \FF 1$, $\vphi_0\inv(\delta) = \delta$ and, hence, using Lemma \ref{lemma:bar-dB}, we have that
    \[
        B = \overline {\overline B} = \overline {\delta B} = (-1)^{|\delta|} \vphi_0\inv(\delta) \overline B= \delta \overline B = \delta^2 B.
    \]
    This implies $\delta^2 = 1$. 
    Therefore $\delta \in \{ \pm 1 \}$, so $\overline B = \pm B$, concluding the proof. 
\end{proof}

% it is clear that $B(u,vd) = B(u,v)d$. By Equation \eqref{eq:sesquilinear-before-B}, the map $B: \U \times \U \to \D$ is a \emph{$\vphi_0$-sesquilinear form}, \ie,
% \[
%     B(ud, v) = (-1)^ {(|B| + |u|)|d|}
% \]

% Consider the map $\vphi_0: \D \to \D$ given by 

% Consider the map $\vphi_0(d) := \sign{\vphi_1}{d}\vphi_1\inv d \vphi_1$ (where juxtaposition denotes composition of maps on the right).
% Then
% \begin{equation}\label{eq:sesquilinear-before-B}
%     (ud)\vphi_1 = (u) (d\vphi_1) =  (u)(\vphi_1\vphi_1\inv d \vphi_1) = \sign{d}{\vphi_1}(u)\vphi_1 \,\vphi_0(d).
% \end{equation}
% It is straightforward to check that $\vphi_0: \D = \End_R(\mc U) \to \End_R(\mc U\Star) = \D\sop$ is an isomorphism of superalgebras.
% We will consider $\vphi_0$ as a super-anti-automorphism on $\D$.

%Recall (Subsection \ref{ssec:superlinear-maps}) that $\vphi_1^\circ$ is an $R$-superlinear map on the left. Using it, the Equation \eqref{eq:sesquilinear-before-B} becomes
% \begin{equation}\label{eq:sesquilinear-with-vphi1-circ}
%     \vphi_1^\circ (ud) = \vphi_1^\circ (u) \vphi_0 (d)
% \end{equation}

%For each $u\in \U$, $\vphi_1(u)$ is an $\D$-linear map on the left. We, then, can define \[B(u,v) = \sign{\vphi_1}{u} ((u)\vphi_1)(v),\]
%or, using the notation introduced in Subsection \ref{ssec:superlinear-maps},
% \[
%     B(u,v) = \vphi_1^\circ (u) (v),
% \]
% for all $u,v \in U$.


%We will consider $\vphi_0$ as a super-anti-automorphism on $\D$.

%Recall (Subsection \ref{ssec:superlinear-maps}) that $\vphi_1^\circ$ is an $R$-superlinear map on the left. Using it, the Equation \eqref{eq:sesquilinear-before-B} becomes


%Since  this new action makes $\mc U\Star$ a simple graded left $R$-supermodule. Also, since we had $\D\sop = \End_{R\sop} (\U\Star)$, we c it follows that $\D\sop$ can be identif
%We, then, have that $\mc U\Star$ is a simple graded left $R$-module and that $\D\sop = \End_R(\mc U\Star)$.

%Since $R$ acts on the left, we will use the convention of writing $R$-superlinear maps on the right. % (see Subsection \ref{ssec:superlinear-maps}).
%By Proposition {\tt ??}, $R$ has only one graded simple supermodule up to isomorphism and shift, \ie, there is an invertible $R$-linear map $\vphi_1: \mc U \to \mc U\Star$ which is homogeneous of some degree $(g_0, \alpha)\in G^\#$.



% Using the notation just introduced, Equation \eqref{eq:sesquilinear-before-B} becomes
% \begin{equation}%\label{eq:sesquilinear-with-vphi1-circ}
%     \vphi_1^\circ (ud) = \vphi_1^\circ (u) \tilde\vphi_0 (d)
% \end{equation}
% or, puting $\vphi_0(d) = (\tilde \vphi_0(d))^\circ$,
% \begin{equation}\label{eq:sesquilinear-with-vphi1-circ}
%     \vphi_1^\circ (ud) = (-1)^{|d|(|\vphi_1|+|u|)}\vphi_0 (d)\vphi_1^\circ (u).
% \end{equation}







% It is important to note that the map $\vphi_1$ is not uniquely determined. For every nonzero homogeneous element $d\in\D$, the map $d\vphi_1: \mc U \to \mc U\Star$ is also an invertible, homogeneous and $R$-superlinear. If we consider $\vphi_1$ fixed, every such map can be obtained this way: if $\vphi_1' : \mc U \to \mc U\Star$ is an invertible homogeneous $R$-superlinear map, then we could take $d := \vphi'\vphi\inv \in \End_R(\mc U) = \D$ and, hence, $\vphi_1' = d \vphi_1$.

% Changing $\vphi_1$ to $\vphi_1'$ also changes the super-anti-automorphism $\vphi_0 (c) = \sign{\vphi_1}{d}\vphi_1\inv d \vphi_1$ to $\vphi_0':= \sign{\delta\vphi_1}{d}\vphi_1\inv\delta\inv d \delta\vphi_1= \sign{\delta}{d} \vphi_0(\delta\inv d \delta)$. In other words, $\vphi_0' = \vphi_0 \circ \operatorname{sInt}_\delta$, where $\operatorname{sInt}_\delta: $

% The choice of the map $\vphi_1$ 

%If $\vphi_1' : \mc U \to \mc U\Star$ is another invertible and homogeneous $R$-superlinear map, then $\vphi'\vphi\inv: \mc U \to \mc U$ is a nonzero homogeneous element of $\D = \End_R(\mc U)$.
%Conversely, for each nonzero homogeneous element $\delta\in\D$, the map $\vphi_1' := \delta\vphi_1: \mc U \to \mc U\Star$ is $R$-superlinear, invertible and homogeneous.
% Changing $\vphi_1$ to $\vphi_1'$ also changes the super-anti-automorphism $\vphi_0(d) = \sign{\vphi_1}{d}\vphi_1\inv d \vphi_1$ to $\vphi_0':= $


% depends on the choice of $\vphi_1$: if we consider $\delta\vphi_1$ instead, then we change $\vphi_0(d) = \vphi_1\inv d \vphi_1$ to $\vphi_0'(d) = \vphi_0(\delta\inv d \delta)$.

\section{Graded division superalgebras with super-anti-automorphism}
% -----------------------------------------

We are now going is to investigate the finite dimensional graded division superalgebras that admit a super-anti-automorphism. 
Throughout this section, we will assume that $\FF$ is an algebraically closed field.
In this case, because of Theorem \ref{thm:vphi-involution-iff-delta-pm-1}, we are primarily interested in the case the graded division superalgebra admits a superinvolution.
Our final goal is to classify the division gradings on finite dimensional superinvolution-simple associative superalgebras.
 
Recall that a graded division superalgebra is the same as a graded division algebra if we consider the $G^\#$-grading. 
In particular, the isomorphism class of a finite dimensional graded division superalgebra $\D$ is determined by a pair $(T, \beta)$ where $T \coloneqq \supp \D \subseteq G^\#$ is a finite abelian group and $\beta\from T\times T \to \FF^\times$ is an alternating bicharacter {\tt (see ??)}. 
Instead of writing subscripts for the $G$-grading and superscripts for the canonical $\ZZ_2$-grading, it will be convenient to write $\D = \bigoplus_{t\in T} \D_t$ and recover the parity via the map $| \cdot |\from T \to \ZZ_2$ which is the restriction of the projection $G^\# = G\times \ZZ_2 \to \ZZ_2$.

Since each component $\D_t$ of $\D$ is one-dimensional, an invertible degree preserving map $\vphi_0\from \D \to \D$ is completely determined by a map $\eta\from T \to \FF^\times$ such that $\vphi_0(X_t) = \eta(t) X_t$ for all $t\in T$ and $X_t\in \D_t$.

\begin{prop}\label{prop:superpolarization}
    Let $\vphi_0\from \D \to \D$ be the invertible degree preserving map determined by $\eta\from T \to \FF^\times$ as above. 
    Then $\vphi_0$ is a super-anti-automorphism if, and only if,
    %
    \begin{equation}\label{eq:superpolarization}
        \forall a,b\in \D, \quad \beta(a,b) = (-1)^{|a||b|} \eta(ab) \eta(a)\inv \eta(b)\inv.
    \end{equation}
    %
    Moreover, if $\D$ admits a super-anti-automorphism, then $\beta$ can only take values $\pm 1$.
\end{prop}

\begin{proof}
    For all $a,b \in T$, let $X_a \in \D_a$ and $X_b\in \D_b$. Then:
    %
    \begin{alignat*}{2}
        &&\vphi_0(X_a X_b) &= (-1)^{|a||b|} \vphi_0(X_b) \vphi_0(X_a)\\
        \iff&&\,\, \eta(ab)X_a X_b &= (-1)^{|a||b|} \eta(a) \eta(b) X_b X_a\\
        \iff&&\, \eta(ab)X_a X_b &= (-1)^{|a||b|} \eta(a) \eta(b) \beta(b,a) X_a X_b\\
        \iff&& \eta(ab) &= (-1)^{|a||b|} \eta(a) \eta(b) \beta(b,a)
        \\
        \iff&& \beta(b, a) &= (-1)^{|a||b|} \eta(ab) \eta(a)\inv \eta(b)\inv.
    \end{alignat*}
    The right-hand side of this last equation does not depend on the order of $a$ and $b$, hence $\beta(b,a) = \beta(a,b)$. 
    Since $\beta$ is alternating, $\beta(b, a) = \beta (a, b)\inv$, therefore $\beta(a,b)^2 = 1$, from where we conclude that $\beta(a,b) = \pm 1$.
\end{proof}

Note that, for the map $\vphi_0$ be involutive, we need $\eta(t)^2 = 1$, for every $t\in T$. 
If we put $a = b = t$ in Equation \eqref{eq:superpolarization}, since $\beta (t, t) = 1$, we get
%
\begin{equation}\label{eq:eta-t-square}
    \eta (t)^2 = (-1)^{|t|} \eta(t^2).
\end{equation}
%
As we are going to see, this implies that we cannot have superinvolutions on a finite dimensional graded division superalgebra that is odd and simple as superalgebra.%  allow us to have In particular, if we want $\vphi_0$ to be a superinvolution, we need $\eta (t^2) = (-1)^{|t|}$, for every $t\in T$.

\begin{lemma}
    Suppose $T = \supp \D$ is an elementary $2$-group and let $\vphi_0\from \D \to \D$ be a super-anti-automorphism. 
    If $\D$ is even, then $\vphi_0$ is a superinvolution, but
    if $\D$ is odd, $\vphi_0$ has order $4$.
\end{lemma}

\begin{proof}
    First of all, note that $\eta (e) = 1$. 
    Let $t\in T$.  
    If $t$ is even, Equation \eqref{eq:eta-t-square} implies $\eta(t)^2 = \eta (t^2) = \eta(e) = 1$, so  $\eta(t) \in \{ \pm 1 \}$. 
    If $\D\odd = 0$, this means $\vphi_0^2 = \id$. 
    Now, if $t$ is odd, Equation \eqref{eq:eta-t-square} implies $\eta(t)^2 = -\eta (t^2) 1 = -1$, hence $\eta(t) \in \{ \pm \sqrt{-1} \}$. 
    Therefore, if $\D\odd \neq 0$, $\vphi_0$ has order $4$.
\end{proof}

The next result can be found in ??, but we will deduce it from the theory we have developed.

\begin{prop}
    The associative superalgebra $Q(n)$ does not admit a superinvolution.
\end{prop}

\begin{proof}
    
\end{proof}

% The map $\vphi_0$ is a super-anti-automorphism if, and only if, for all $a,b \in T$, $X_a \in \D_a$ and $X_b\in \D_b$,
%
% \begin{align*}
%     &\vphi_0(X_a X_b) = (-1)^{|a||b|} \vphi_0(X_b) \vphi_0(X_a)\\ \iff& \eta(ab)X_a X_b = (-1)^{|a||b|} \eta(a) \eta(b) X_b X_a\\ \iff& \eta(ab)X_a X_b = (-1)^{|a||b|} \eta(a) \eta(b) \beta(b,a) X_a X_b\\ \iff& \eta(ab) = (-1)^{|a||b|} \eta(a) \eta(b) \beta(b,a), \numberthis \label{eq:eta-super-anti-auto}
% \end{align*}
%

We are now going to consider the cases where $(\D, \vphi_0)$ is a simple algebra with super-anti-automorphism.

\begin{prop}
    If $\D$ is simple as an algebra, then $T$ is an elementary $2$-group.
\end{prop}

\begin{proof}
    If $\D$ simple as algebra, then $\beta$ nondegenerate, so the map $T \to \widehat T$ given by $t \mapsto \beta(t, \cdot)$ is a group isomorphism.
    In particular, if $t\in T$ has order $n>2$, then $\beta(t, \cdot)\in \widehat T$ also has order $n$.
    But, by Proposition \ref{prop:}, $\beta$ only takes values on $\{ \pm 1\}$, so  $\beta(t, \cdot )^2 =1$ and, hence, the order of $t$ is at most $2$.
\end{proof}

\begin{cor}\label{cor:super-anti-order-4}
    Suppose $\D$ is simple as an algebra.
    If $\D\odd = 0$, every super-anti-automorphism is a superinvolution, but if $\D\odd \neq 0$, every super-anti-automorphism has order $4$.
\end{cor}

\begin{prop}
    Suppose $T = \supp \D$ is an elementary $2$-group. If $\D$ is even, then every super-anti-automorphism is a superinvolution, but
    if $\D$ is odd, then every super-anti-automorphism has order $4$.
\end{prop}



As a consequence of Corollary \ref{cor:super-anti-order-4}, all Type I gradings on the superalgebras in the series $B$, $C$, $D$ and $P$ are even.

We will now focus on the case $\D$ is odd, i.e., $\D\odd \neq 0$.

\begin{ex}\label{ex:supertransp-graded}
    Consider on $M(1,1)$ the $\ZZ_2$-grading given by declaring $\begin{pmatrix}
       1 & 0 \\
       0 & 1
     \end{pmatrix}$ and
     $\begin{pmatrix}
       0 & i \\
       1 & 0
     \end{pmatrix}$ to have degree $\overline 0$ and
     $\begin{pmatrix}
       -1 & 0 \\
       0 & 1
     \end{pmatrix}$ and
     $\begin{pmatrix}
       0 & -i \\
       1 & 0
     \end{pmatrix}$ to have degree $\overline 1$. In other words, as a $\ZZ_2^\#$-grading on $M_2(\FF)$ we have\\
     %
     \begin{center}
     \begin{tabular}{ l c r }
     $\deg \begin{pmatrix}
      \phantom{-}1 & 0\phantom{..} \\
      \phantom{-}0 & 1\phantom{..}
     \end{pmatrix} = (\bar 0, \bar 0)$, && $\deg \begin{pmatrix}
      \phantom{.}0 & \phantom{-}i\phantom{.} \\
      \phantom{.}1 & \phantom{-}0\phantom{.}
     \end{pmatrix} = (\bar 0, \bar 1)$,\\
     $\deg \begin{pmatrix}
      -1 & 0\phantom{..} \\
       \phantom{-}0 & 1\phantom{..}
     \end{pmatrix} = (\bar 1, \bar 0)$ &
     and
     & $\deg \begin{pmatrix}
      \phantom{.}0 & -i\phantom{.} \\
      \phantom{.}1 & \phantom{-}0\phantom{.}
     \end{pmatrix} = (\bar 1, \bar 1)$,
     \end{tabular}
     \end{center}
     %
     which is clearly a division grading. The supertranspose is a super-anti-automorphism on it.
\end{ex}

\begin{ex}\label{ex:FZZ_4-with-superinvoltion}
    Let $T= \langle \omega \rangle$ where $\omega$ has order $4$, $\beta$ be trivial and $| \cdot |: T \to \ZZ_2$ be the only nontrivial homomorphism. In other words, $\D = \FF T$ with $\D\even = \FF 1 \oplus F \omega^2$ and $\D\odd = \FF \omega \oplus \omega^3$. We define a superivolution by $\eta (1) = 1$, $\eta (\omega) = 1$, $\eta (\omega^2) = -1$ and $\eta(\omega^3) = -1$. %We will denote this superinvolution by 
    
    One could verify by hand that $\eta$ satisfies Equation \eqref{eq:eta-super-anti-auto}, but it also follows from the following argument. In both $\D$ and $\D\sop$, $\omega$ is an element with order $4$ and both are superalgebras with dimension $4$. Hence, both homomorphisms $\theta: \frac{\FF[x]}{\langle x^4 - 1 \rangle} \to \D$ and $\theta': \frac{\FF[x]}{\langle x^4 - 1 \rangle} \to \D\sop$ sending $x$ to $\omega$ are isomorphisms (note that $\theta(x^2) = \omega^2$ and $\theta'(x^2) = (\omega\sop)^2 = - \omega^2$, so they are different maps!). The map $\theta'\circ \theta\inv: \D \to \D\sop$ corresponds to an super-anti-automorphism with the $\eta$ above.
\end{ex}

Example \ref{ex:FZZ_4-with-superinvoltion} is our first example of odd graded division superalgebra with superinvolution. We can use it to transform any even graded division superalgebra into an odd one by ``extending scalars''.

\begin{lemma}
    Let $(R, \vphi)$ and $(S, \psi)$ be graded superalgebras with superinvolution, with grading groups $G$ and $H$, respectively. Then $(R\tensor S, \vphi\tensor \psi)$ is a $G\times H$-graded superalgebra with superinvolution.\qed
\end{lemma}

\begin{lemma}
    Let $\D$ and $\mc E$ be graded division superalgebras, with grading groups $G$ and $H$, respectively. If $\D$ is even, then $\D \tensor \mc E$ is a $G\times H$-graded division superalgebra.
\end{lemma}

\begin{proof}
    It is straightforward that the tensor product of graded division algebras is a graded division algebra, hence the result follows from considering $\D$ as a $G$-graded algebra and $\mc E$ as a $H\times \ZZ_2$-graded algebra.
\end{proof}

WE REMIND THE AUTHOR TO EXPLAIN THE DIFFERENT TYPES OF TENSOR PRODUCT IN THE GENERALITIES.

Using the two Lemmas above and Example \ref{ex:FZZ_4-with-superinvoltion}, we have an infinitely many examples of odd graded division superalgebras. Nevertheless, their centers include a $4$-dimensional superalgebra, so none of them is a Type II odd graded division algebra. 
\begin{ex}
    Consider $M_2(\FF)$ with the division grading
\end{ex}

Neither Example \ref{ex:supertransp-graded} or Example \ref{ex:FZZ_4-with-superinvoltion} appear as graded division superalgebras of a Type II grading on $M(n,n)$. But both can be used to construct such superalgebras. We will now show two constructions, one based on Example \ref{ex:supertransp-graded} and other on Example \ref{ex:FZZ_4-with-superinvoltion}.

% Example \ref{ex:FZZ_4-with-superinvoltion} suggests that to have a graded division superalgebra we need to ``increase'' the canonical $\ZZ_2$-grading to a $\ZZ_4$-grading. We have seen two constructions that do that on Section \ref{sec:loop-and-induced-algebras}.% We will start with

Neither Example \ref{ex:supertransp-graded} or Example \ref{ex:FZZ_4-with-superinvoltion} are examples of odd Type II gradings. Nevertheless, we will now present two constructions of such gradings, one based on Example \ref{ex:supertransp-graded} and other based on Example \ref{ex:FZZ_4-with-superinvoltion}. 

\begin{prop}\label{prop:involution-on-cartesian-product}
    Let $R$ be a $G$-graded superalgebra and $\psi: R \to R$ be a super-anti-automorphism (homogeneous of degree $e$). Then $\vphi: R\times R \to R\times R$ geiven by $\vphi(x, y) = (\vphi\inv (y), \vphi (x))$ is a superinvoltion on the $G$-graded superalgebra $R\times R$. \qed
\end{prop}

Using Proposition \ref{prop:involution-on-cartesian-product} with $R$ being the graded superalgebra of Example \ref{ex:supertransp-graded}, we get a graded superalgebra with involution which is not a graded division superalgebra. To make its grading a division grading, we need to refine it. We could do it explicitly, but to not repeat ourselves we will follow another approach.

\begin{ex}
    Let $(R, \psi)$ be the graded superalgebra with super-anti-automorphism of Example \ref{ex:supertransp-graded}. Interpreting its canonical $\ZZ_2$-grading as a $\frac{\ZZ_4}{\langle \bar 2 \rangle}$-grading, its induced algebra $\D = \operatorname{Ind} x$ is a graded division superalgebra. Indeed, $\D = \chi_1 \tensor R \oplus \chi_i \tensor R$ where $\chi_z: \ZZ_4 \to \FF^\times$ denotes the character that sends $\bar 1$ to $z$. Its grading is $\ldots$ .
    
    Now let $\vphi'$ be the homogeneous of degree $e$ map on induced from $\psi$ and $\theta$ be the automorphism given by the action of $\chi_i$. We define $\vphi := \theta \vphi'$. One can readly check that it a superinvolution on $\D$.
\end{ex}

We will now construct a different model of the same superalgebra. This new construction, though, can be done over any field, even if it does not have a square root of $-1$.

%As we saw in Section \ref{ainda-nao-existe}, in an algebraically closed field of characteristic $0$, the induced algebra is always isomorphic to a corresponding loop algebra.

\begin{prop}
    Let $(S, \psi)$ be the superalgebra with superinvolution of Example \ref{ex:FZZ_4-with-superinvoltion} and let $M_2(\FF)$ be the matrix algebra endowed with the $\ZZ_2 \times \ZZ_2$-division grading given by:...
    
    and let $\theta$ be the transposition on it.
\end{prop}


\section{Lost and found}

We will investigate one last question...

\begin{cor}
    Under the conditions of Theorem \ref{thm:vphi-involution-iff-delta-pm-1}, assume further that $\D$ is not an even 
\end{cor}

We have that $\overline B = \delta B $ where $\delta \in \{\pm 1\}$. 
Changing $B$ for $B' \coloneqq dB$, by Lemma \ref{lemma:bar-dB}, we get $\overline{B'} = (-1)^{|d|} \vphi_0\inv (d) \overline B = (-1)^{|d|} \vphi_0\inv (d) \delta B = (-1)^{|d|} \vphi_0\inv (d) \delta d\inv d B = (-1)^{|d|} \vphi_0\inv (d) \delta d\inv \overline B = (-1)^{|d|} \vphi_0\inv (d) d\inv \delta \overline B$.
So we want to make the scalar $(-1)^{|d|} \vphi_0\inv (d) d\inv$ be $-1$. 
Note that $(-1)^{|d|} \vphi_0\inv (d) d\inv = -1$ iff $\vphi_0\inv (d) = - (-1)^{|d|} d$. 
We cannot find such delta iff $ (-1)^{|d|} \vphi_0 = \id$ (see 1st Prop in next section), which implies $\D$ supercommutative. But not odd graded division superalgebra can be supercommutative (since it implies $x^2 = -x^2$ for odd $x$, hence we cannot have invertible odd elements). So the only problem is when $\D$ is even and commutative. 


% \bibliographystyle{amsalpha}
% \bibliography{bibliography}

\end{document}